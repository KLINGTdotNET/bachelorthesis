\chapter*{Abstract}

% todo: Abstract schreiben und schlüsselwörter vervollständigen

% Motivation des Text
% Fragestellung
% Methodologie
% Ergebnisse
% Implikationen, Schlussfolgerungen

\section*{German}

Die vorliegende Arbeit beschreibt den Entwurf eines Codegenerators mit austauschbarer Zielsprache, der aus der abstrakten Beschreibung der Spreadshirt-\textsc{Api} eine Client-Bibliothek generieren soll. Die Implementierung des Codegenerators erfolgt in Java, die Zielsprache der Bibliothekt ist \textsc{Php}. 

Es werden die Grundlagen von WebServices, Codegeneratoren und Dokumentbeschreibungssprachen erläutert und darauf aufbauend Datenmodelle erstellt welche die Beschreibung der Spreadshirt-\textsc{Api} für den Generator enthalten. Außerdem wird das erstellte Sprachenmodell betrachtet, welches die Konstrukte der zu erzeugenden Zielsprache kapselt. Aufbau und Ablauf eines Codegeneratorsystems werden ebenfalls beschrieben.

Die durch den Codegenerator erstellte Client-Bibliothek wird anhand eines Anwendungsbeispieles evaluiert.

\section*{English}

The present thesis describes the design of a codegenerator with exchangeable target-language, that generates a client-library from the abstract description of the Spreadshirt-\textsc{Api}. The implementation of the codegenerator is made in Java, the target-language of the client-library is \textsc{Php}.

The basics of web services, codegenerators and document-description-languages will be explained and based on that datamodels will be created, that contains the description of Spreadshirt-\textsc{Api}, for the generator. Furthermore the created language-model will be examined, which encapsulates the constructs of the target-language that will be generated. Structure and process of a codegeneratorsystem will be described, as well.

The client-library that was created by the codegenerator will be evaluated on the basis of a usage example.

\section*{Keywords}

Codegeneration, \textsc{Rest}ful Web Service, Modeling, Client-Library, Spreadshirt-\textsc{Api}, Polyglot
\newpage
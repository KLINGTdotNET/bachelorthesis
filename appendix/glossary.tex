% \newglossaryentry{tag}{name=foo, description={bar}}
%
% Note: Paragraphs are not allowed
%

\newglossaryentry{REST}{
	name=REST, 
	description={
		\emph{Representational State Transfer} (deutsch: \enquote{Gegenständlicher Zustandstransfer}) ist ein Softwarearchitekturstil für Webanwendungen, welcher von Roy Fielding in seiner \printhref{http://www.ics.uci.edu/~fielding/pubs/dissertation/fielding_dissertation.pdf}{Dissertation} beschrieben wurde. Die Daten liegen dabei in eindeutig addressierbaren \emph{resources} vor. Die Interaktion basiert auf dem Austausch von \emph{representations} -- also ein Dokument was den aktuellen oder gewünschten Zustand einer resource beschreibt.
		Beispiel-URL für das Item \emph{84} aus dem Warenkorb \emph{42}:\\
		\texttt{http://api.spreadshirt.net/api/v1/baskets/84/item/42}
	}
}

\newglossaryentry{RESTful}{
	name=RESTful,
	description={
		Als \emph{RESTful} bezeichnet man einen Webservice der den Prinzipien von REST entspricht
	},
	see=REST
}

\newglossaryentry{API}{
	name=API, 
	description={
		\emph{Application Programming Interface} (deutsch: \enquote{Schnittstelle zur Anwendungsprogrammierung}) spezifiziert wie Softwarekomponenten über diese Schnittstelle miteinander interagieren können
	}
}

\newglossaryentry{XML}{
	name=XML,
	description={
		\emph{Extensible Markup Language} (deutsch: \enquote{erweiterbare Auszeichnungssprache}) ist ein Mensch- und Maschinenlesbares Format für Codierung und Austausch von Daten, \printhref{http://www.w3.org/TR/REC-xml}{spezifiziert vom W3C}
	}
}

\newglossaryentry{JSON}{
	name=JSON,
	description={
		\emph{JavaScript Object Notation} ist ein Mensch- und Maschinenlesbares Format zu Codierung und Austausch von Daten. Bietet im Gegensatz zu XML keine Erweiterbarkeit und Unterstützung für Namesräume, ist aber kompakter und einfacher zu parsen
	}, 
	see=XML
}

\newglossaryentry{XSD}{
	name=XSD,
	description={
		\emph{XML Schema Description} enthält Regeln für den Aufbau und zum Validieren einer XML-Datei
	},
	see=XML
}

\newglossaryentry{WADL}{
	name=WADL,
	description={
		\emph{Web Application Description Language} ist eine maschinenlesbare Beschreibung einer HTTP-basierten Webanwendung
	},
	see=XML
}

\newglossaryentry{Polyglot}{
	name=Polyglot,
	description={
		\emph{mehrsprachig}
	}
}

\newglossaryentry{Metaprogramming}{
	name=Metaprogramming,
	description={
		beschreibt das erstellen von Programmen welche sich selbst, oder andere Programme, modifizieren oder die einen Teil des Kompilierungsschrittes übernehmen (bspw. der C-Präprozessor)
	}
}

\newglossaryentry{DSL}{
	name=DSL,
	description={
		\emph{Domain Specific Language} (deutsch: \enquote{Domänenspezifische Sprache}) ist eine Programmiersprache die nur auf eine bestimmte Domäne oder auch Problembereich optimiert ist.
	}
}

\newglossaryentry{template-engine}{
	name=Template-Engine,
	description={
		Eine \emph{Template-Engine} ersetzt markierte Bereiche in einer Template-Datei (i. Allg. Textdateien) nach vorgegebenen Regeln
	}
}

\newglossaryentry{MIME}{
	name=MIME,
	description={
		\emph{Multipurpose Internet Mail Extensions} dienen zu Deklaration von Inhalten (Typ des Inhalts) in verschiedenen Internetprotokollen.
	}
}

\newglossaryentry{URI}{
	name=URI,
	description={
		\emph{Unified Resource Identifier} ist ein Folge von Zeichen, die einen Name oder eine Web-Ressource identifiziert.
	}
}

\newglossaryentry{URL}{
	name=URL,
	description={
		\emph{Unified Resource Locator} sind eine Untermenge der \emph{URIs}. Der Unterschied besteht in der expliziten Angabe des Zugrissmechanismus und des Ortes (\enquote{Location}) durch \emph{URLs}, bspw. \texttt{http} oder \texttt{ftp}
	},
	see=URI
}

\newglossaryentry{DTD}{
	name=DTD,
	description={
		\emph{Document Type Definition}, manchmal auch \emph{Data Type Definition}, ist eine Menge von Angaben die einen Dokumenttyp beschreiben. Es werden konkret Element- und Attributtypen, Entitäten und deren Struktur beschrieben. Die bekanntesten Schemasprachen für XML-Dokumente sind XSD und RelaxNG
	},
	see=XSD
}
\section{XML-Parser}
\label{sec:xml_parser}
%todo: quellen

Um mit der abstrakten Beschreibung der Spreadshirt-API arbeiten zu können, muss diese zuerst in das interne Datenmodell überführt werden. Diese liegt in XML-basierter Form vor, welche in \cref{sec:web_services} näher beschrieben wurde. Folglich wird ein XML-Parser für die Verarbeitung der Beschreibungsformate benötigt.  

Die \emph{Java API for XML Processing} kurz \emph{JAXP} abstrahiert die Parserschnittstelle von der eigentlichen Implementierung. JAXP ist dabei keine einzelne API sondern es beschreibt Schnittstellen für folgende vier XML-Parser Modelle:

\begin{description}
    \item[DOM] \emph{Document Object Model}-Parser überführen das XML-Dokument in ein baumartiges Objektmodell, welches vollständig im Arbeitsspeicher liegt.
    \item[SAX] \emph{Simple API for XML} basierte, sogenannte Push-Parser verarbeiten das XML-Dokument seriell und eventbasiert. Ein Event ist hierbei bspw. ein öffnendes oder schließendes XML-Element.
    \item[StAX] \emph{Streaming API for XML} basierte, sogenannte Pull-Parser arbeiten ebenso wie bei \emph{SAX} seriell und eventbasiert, können aber im Gegensatz dazu die Erzeugung von Events selber steuern. 
    \item[TrAX] \emph{Transformation API for XML} bietet eine Schnittstelle mit der sich XML-Dokumente durch \emph{Extensible Stylesheet Language Transformations (XSLT)} in Java transformieren lassen.
\end{description}

\Cref{tab:xmlParsingModels} enthält eine Übersicht zu den Parsing-Konzepten, ausgenommen \emph{TrAX} da diese API vorwiegend für die Modifikation von XML-Dateien gedacht ist.

Bei dem zu entwickelnden Codegenerator sind der Speicherverbrauch und die verwendete CPU-Zeit kein Teil der \emph{nichtfunktionalen Anforderungen}, somit fiel die Entscheidung auf einen DOM-Parser. Dieser lässt sich durch das komplett im Speicher gehaltene Objektmodell mit geringem Aufwand verwenden. Durch JAXP ist die Implementierung transparent und es wird die im \emph{JDK} enthaltene Standart DOM-Parser Implementierung verwendet.

\begin{table}[tb]
    \begin{longtable}[c]{r l l l}
        \toprule
        \rowcolor{lightgray}
        & \textbf{DOM}   & \textbf{SAX}   & \textbf{StAX} \\
        \midrule
        \textbf{API-Typ}                    & In-Memory Tree    & push-streaming    & pull-streaming\\
        \textbf{Speicherverbrauch}          & hoch              & gering            & \textless{} DOM\\
        \textbf{Prozessorlast}              & hoch              & gering            & gering \\
        \textbf{Elementzugriff}             & beliebig          & seriell           & seriell \\
        \textbf{Nutzerfreundlichkeit}       & niedrig           & hoch              & mittel \\
        \textbf{XML schreiben}              & ja                & nein              & ja \\
        \bottomrule
        \caption{Übersicht über die verschiedenen XML-Parsing Konzepte in JAXP}
        \label{tab:xmlParsingModels}
    \end{longtable}
\end{table}

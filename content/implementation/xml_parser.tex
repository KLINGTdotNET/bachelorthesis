\section{XML-Parser}
\label{sec:xml_parser}
%todo: quellen

Um mit der abstrakten Beschreibung der Spreadshirt-\gls{API} arbeiten zu können, muss diese zuerst in das interne Datenmodell überführt werden. Diese liegt in \gls{XML}-basierter Form vor, welche in \cref{chap:web_services} näher beschrieben wurde. Folglich wird ein \gls{XML}-Parser für die Verarbeitung der Beschreibungsformate benötigt.  

Die \enquote{Java \gls{API} for \gls{XML} Processing} kurz \textsc{Jaxp} abstrahiert die Parserschnittstelle von der eigentlichen Implementierung. \textsc{Jaxp} ist dabei keine einzelne \gls{API}, sondern es beschreibt Schnittstellen für folgende vier \gls{XML}-Parser Modelle:

\begin{description}
    \item[\textsc{Dom}] \enquote{Document Object Model}-Parser überführen das \gls{XML}-Dokument in ein baumartiges Objektmodell, welches vollständig im Arbeitsspeicher liegt.
    \item[\textsc{Sax}] Die \enquote{Simple \gls{API} for \gls{XML}} basierten sogenannten Push-Parser verarbeiten das \gls{XML}-Dokument seriell und eventbasiert. Ein Event ist hierbei beispielsweise ein öffnendes oder schließendes Xml-Element.
    \item[StAX] Auf \enquote{Streaming \gls{API} for \gls{XML}} basierende sogenannte Pull-Parser arbeiten ebenso wie bei \textsc{Sax} seriell und eventbasiert, können aber im Gegensatz dazu die Erzeugung von Events selber steuern. 
    \item[TrAX] Die \enquote{Transformation \gls{API} for \gls{XML}} bietet eine Schnittstelle, mit der sich \gls{XML}-Dokumente durch \emph{Extensible Stylesheet Language Transformations (\textsc{Xslt})} in Java transformieren lassen.
\end{description}

\Cref{tab:xmlParsingModels} enthält eine Übersicht zu den Parsing-Konzepten, ausgenommen TrAX, da diese \gls{API} vorwiegend für die Modifikation von \gls{XML}-Dateien gedacht ist.

Bei dem zu entwickelnden Codegenerator sind der Speicherverbrauch und die verwendete \textsc{Cpu}-Zeit kein Teil der \emph{nichtfunktionalen Anforderungen}. Somit fiel die Entscheidung auf einen \textsc{Dom}-Parser. Dieser lässt sich durch das komplett im Speicher gehaltene Objektmodell mit geringem Aufwand verwenden. Durch \textsc{Jaxp} ist die Implementierung transparent und es wird die im \textsc{Jdk} enthaltene Standart \textsc{Dom}-Parser Implementierung verwendet.

\begin{table}[hb]
    \begin{longtable}[c]{r l l l}
        \toprule
        \rowcolor{lightgray}
        & \textbf{DOM}   & \textbf{SAX}   & \textbf{StAX} \\
        \midrule
        \textbf{API-Typ}                    & In-Memory Tree    & push-streaming    & pull-streaming\\
        \textbf{Speicherverbrauch}          & hoch              & gering            & \textless{} \textsc{Dom}\\
        \textbf{Prozessorlast}              & hoch              & gering            & gering \\
        \textbf{Elementzugriff}             & beliebig          & seriell           & seriell \\
        \textbf{Nutzerfreundlichkeit}       & niedrig           & hoch              & mittel \\
        \textbf{XML schreiben}              & ja                & nein              & ja \\
        \bottomrule
        \caption{Übersicht über die verschiedenen \gls{XML}-Parsing Konzepte in \textsc{Jaxp}}
        \label{tab:xmlParsingModels}
    \end{longtable}
\end{table}

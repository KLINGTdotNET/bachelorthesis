\chapter{Evaluation}
\label{chap:evaluation}

\chapterQuote{There are two ways to write error-free programs; only the third one works}{\citeauthor{alanPerlis}}{\citeyear{alanPerlis}}{\cite{alanPerlis}}

Die Bewertung --- oder auch Evaluierung --- der generierten Client-Bibliothek gegenüber den Anforderungen (siehe \cref{item:requirements}) anhand von Anwendungsbeispielen ist Bestandteil dieses Kapitels.

% todo: Überschrift ändern
\section{PHP-Zielsprachenmodell}\label{sec:php_target_language_model}

Das PHP-Zielsprachemodell ist die Implementierung der Schnittstellen die durch das Sprachenmodell vorgegeben sind.

\Cref{fig:modelRepresentationOfBatchDTO} zeigt die Gegenüberstellung von \Cref{lst:batchDTO} in Form des \gls{AST} der durch das Sprachmodell gebildet wird. 

\begin{minipage}{\textwidth}
\begin{lstlisting}[
    language=PHP,
    caption=Ausschnitt der Datenklasse BatchDTO,
    label=lst:batchDTO
]
<?php
   require_once('Dto.php');
   require_once('OperationDTO.php');

   class BatchDTO
   {
      private $operations = array(); // operationDTO 
      ...
      public static function fromXML(
            /* SimpleXMLElement */ $xml
         )
      {
         $operations = OperationDTO::fromXML(/* SimpleXMLElement */ $xml->operations)
;          ...
      }
      ...
   }
?>
\end{lstlisting}
\end{minipage}

\begin{sidewaysfigure}
    \centering
    \resizebox{\textwidth}{!}{
      \begin{tikzpicture}[every tree node/.style={font=\ttfamily}]    
    \Tree
    [.ClassFile 
        [ .Import 
            [ .Literal 
                [ .PrimitiveType 
                    {'Dto.php'}
                ] 
            ]            
        ]
        [ .Import 
            {\ldots}
        ]    
        [ .CommonClass
            {'BatchDTO'}
            [
                .AssignmentStatement
                [ .Variable
                    {"operations"}
                    [ .Modifiers
                        {"private"}
                    ]
                ]
                [ .Operator 
                    {"="}
                ]                
                [
                    .MethodInvocation
                    {"array"}
                ]
            ]
            [
                .DefinitionStatement        
                {"fromXML"}
                [
                    .Modifiers
                    {"public"}
                    {"static"}
                ] 
                [ .Identifier
                    [ .Comment
                        {"SimpleXMLElement"}
                    ]
                    {"xml"}
                ]
                [
                    .Block
                    [
                        [ .AssignmentStatement 
                            [ .Identifier
                                {"operations"}
                            ]
                            [ .Operator
                                {"="}
                            ]
                            [ .SimpleExpression
                                [ .Identifier
                                    {"OperationDTO"}
                                ]                
                                [ .Operator 
                                    {"::"}
                                ]
                                [
                                    .MethodInvocation
                                    {"fromXML"}
                                    [                                                
                                        .SimpleExpression                                                
                                        [ .Identifier
                                            [ .Comment
                                                {"SimpleXMLElement"}
                                            ]
                                            {"xml"}
                                        ]
                                        [ .Operator
                                            {"->"}
                                        ]
                                        [ .Identifier   {"operations"}
                                        ]
                                    ]
                                ]
                            ]
                        ]                                                    
                    ]
                ]
            ]            
        ]
    ]
\end{tikzpicture}
    }
    \caption{Darstellung von BatchDTO aus \Cref{lst:batchDTO} im Sprachenmodell [\textbf{Klasse}, \emph{Zeichenkette}]}
    \label{fig:modelRepresentationOfBatchDTO}
\end{sidewaysfigure}

\section{Nutzbarkeit}
\label{sec:usability}

Quellcodefiles welche Bibliotheksfunktionalitäten nutzen müssen im Wurzelverzeichnis selbiger liegen. Dies ist damit begründet das der \textsc{Php}-Interpreter bei Importanweisungen mit relativen Pfaden, den aktuellen Pfad der ausgeführten Datei als Basis nimmt.

Derzeit ist die Bibliothek noch eingeschränkt nutzbar, da die De-/Serialisier von strukturierten Typen noch nicht fehlerfrei generiert werden. Die Informationen, die nötig sind um Datenklassen verlustfrei zu serialisieren beziehungsweise deren \gls{XML}-Repräsentation zu deserialisieren sind im Schema-Modell (siehe \cref{sec:schema_model}) vorhanden, der Algorithmus im Codegenerator zur Erzeugung dieser Methoden muss deshalb überarbeitet werden.

\section{Leistungsbewertung}
\label{sec:performance_measurement}

Im folgenden soll eine typischer Arbeitsablauf mit der Client-Bibliothek gezeigt werden. Der Client sollte zum Einstieg ein Objekt der Datenklasse \textbf{LoginDTO} anlegen \ding{202} und seine Login-Informationen eintragen, dies wird benötigt um über die Ressourcenklasse \textbf{Sessions} auf der gleichnamigen \gls{API}-Ressource eine neue Sitzung (\enquote{Session}) anzulegen.
Die \emph{Response} der Session-Ressource enthält eine \gls{URI} auf die Ressource der angelegten Sitzung. Diese \gls{URI} enthält die \emph{SessionID} welche durch \ding{204} aus der \gls{URI} extrahiert wird.

Nun sind alle nötigen Informationen vorhanden um ein Objekt der Klasse \textbf{ApiUser} anzulegen \ding{205}, welches die Authentifizierungsinformationen für gesicherte Ressourcen der Spreadshirt-\gls{API} enthält.

Nun kann über die Bibliotheksklassen auch auf geschützte \gls{API}-Ressourcen zugegriffen werden, in diesem Fall werden beispielsweise die Produkte eines bestimmten Users abgefragt. Entegegen den vorherigen Aufrufen von Methoden der Ressourceklassen wird ein Parameter-Array erzeugt \ding{206} um den \emph{mediaType} der Response festzulegen.
Vor dem Zugriff auf die Ressource muss noch die entsprechende Ressourcenklasse instanziiert werden \ding{207}.

\begin{minipage}{\textwidth}
\begin{lstlisting}[
    language=PHP,
    caption=Beispiel für eine Interaktion mit der Spreadshirt-API über die generierte Client-Bibliothek (Authentifizierungsinformationen wurden anonymisiert),
    label=lst:giveMeALabel
]
<?php

require_once("data/LoginDTO.php");
require_once("Static/apiUser.php");
require_once("resources/UsersUserId.php");
require_once("resources/UsersUserIdProducts.php");
require_once("resources/Sessions.php");

$loginDTO = new LoginDTO(); //@\ding{202}@//
$loginDTO->setUsername("username");
$loginDTO->setPassword("password");
$resource = new Sessions();
$session = $resource->POST(null, $loginDTO); //@\ding{203}@//
$sessionId = preg_replace('/.*\//',"",$session['header']['Location']); //@\ding{204}@//

$apiUser = new ApiUser("userId", "apiKey", "secret", $sessionId); //@\ding{205}@//

$parameters = array("mediaType"=>"json"); //@\ding{206}@//
$resource = new UsersUserIdProducts($userId); //@\ding{207}@//
$products = $resource->GET($parameters, $apiUser); //@\ding{208}@//

?>
\end{lstlisting}
\end{minipage}

%\section{Codemetriken}
%\label{sec:codemetrics}
\section{Codegenerator}
\label{sec:codegenerator}

Nachdem in \cref{sec:concrete_models} die Datenmodelle des Generators betrachtet wurden, widmet sich dieser Abschnitt nur dem Aufbau des Codegenerators und den dort verwendeten Entwurfsmustern, \emph{Factory} (\cref{sec:language_factory}) und \emph{Visitor} (\cref{sec:language_visitor}).

Die Aufgabe des Generators ist die Transformierung des Applikationsmodells in das Modell der Zielsprache. Als Form wurde die des Tier-Generators gewählt, da vollständiger Code vom Generator erzeugt werden soll der keiner Nacharbeit mehr Bedarf. 

\subsection{Language Factory}
\label{sec:language_factory}

Das \emph{Factory}-Pattern behandelt das Problem Familien von Objekten erzeugen zu wollen ohne die konkreten Klassen zu spezifizieren, sondern nur Interfaces festzulegen \cite[][S. 26]{patternsKompakt}.
Um eine Zielsprachenunabhängigkeit zu erreichen, wird dem Generator bei der Erzeugung eine Factory übergeben die das Interface \emph{Language Factory} des Sprachenmodells implementiert. 

Der Generator erzeugt Sprachelemente nur über diese Factory. Ein Aufruf einer Factorymethode gibt ein Sprachelement der Zielsprache zurück, der Generator kennt aber nur den Interface-Typ. Für ihn ist die konkrete Implementierung somit transparent. 
Die \emph{Language Factory} bildet damit die Schnittstelle zwischen dem Generator und der Implementierung des Zielsprachenmodells.

\subsubsection{Language Visitor}
\label{sec:language_visitor}

\citeauthor{patternsKompakt} definieren den Verwenduszweck des Patterns in \cite[][S. 60]{patternsKompakt} so: 
\thesisquote{Dieses Pattern ermöglicht es, neue Operationen auf den Elementen einer
Struktur zu definieren, ohne die Elemente selbst anzupassen.}

Die Aufgabe des \emph{Language Visitor} im Generator ist die Transformation des Sprachenmodells in eine Zeichenketten-Darstellung. Wie in \cref{sec:language_model} schon erwähnt, enthält die Klasse, die das \emph{LanguageVisitor}-Interface implementiert, Regeln für eine syntaktisch korrekte Ausgabe des Sprachenmodells. Zusätzlich können in den LanguageVisitor \enquote{code conventions} implementiert werden, beispielsweise Einrückungstiefen, Zeilenlängen etc.

\subsection{Ausgabemodul}
\label{sec:printer_module}

Das Ausgabemodul beinhaltet Methoden zur Speicherung der Zeichenkettendarstellung aus dem \emph{Language Visitor}. Üblicherweise ist dies die Speicherung in Dateiform, es ist aber ebenso die Ausgabe auf \texttt{stdout} oder die Speicherung in einer Datenbank möglich.

\subsection{Generatorablauf}
\label{sec:generation_process}

Die \Cref{fig:generation_process} stellt den Prozess der Bibliotheksgenerierung in Diagrammform dar.

Zu Beginn steht die Analyse der \gls{API}- und Schema-Beschreibung sowie die Überführung in die entsprechenden Datenmodelle. In einem nachgelagerten Schritt werden diese beiden Datenmodelle zu dem \emph{Applikationsmodell} zusammengeführt. Der Generator erhält dieses Modell und eine \emph{LanguageFactory} als Eingabe. Daraus wird das \emph{Zielsprachenmodell} generiert, welches den zu erzeugenden Code in Form eines \gls{AST} enthält. Aus diesem Modell erzeugt das Ausgabemodul mit Hilfe des \emph{LanguageVisitor} die Bibliotheksdateien. Der \emph{LanguageVisitor} enthält die syntaktischen Informationen, um aus dem \gls{AST} gültigen Code zu erzeugen.

\begin{sidewaysfigure}
    \centering
    \resizebox{ \textwidth}{!}{
        \begin{tikzpicture}[
        node distance=12mm and 8mm,
        every node/.style={font=\scriptsize}
    ]
    % Blocks
    \node(abstractDescription)[greyBlock]{Abstrakte\\Beschreibung\\der Spreadshirt-API};
    \node(dummy1)[dummy, right=of abstractDescription]{};
    \node(wadlAnalysis)[greyBlock, above right=of abstractDescription]{Analyse\\WADL-Datei};
    \node(restModel)[greyBlock, right=of wadlAnalysis]{REST-\\Modell};
    \node(xsdAnalysis)[greyBlock, below right=of abstractDescription]{Analyse\\XSD-Datei};
    \node(schemaModel)[greyBlock, right=of xsdAnalysis]{Schema-\\Modell};
    \node(modelCombine)[greyBlock, right=of dummy1]{Kombinierer};
    \node(applicationModel)[greyBlock, right=of modelCombine]{Applikations-\\Modell};
    \node(generator)[greyBlock, right=of applicationModel]{Generator};
    \node(languageModel)[greyBlock, right=of generator]{Zielsprachen-\\Modell};
    \node(languageFactory)[greyBlock, below=of generator]{Language Factory};
    \node(filePrinter)[greyBlock, right=of languageModel]{File Printer};
    \node(library)[greyBlock, double copy shadow, right=of filePrinter]{Bibliotheks-\\Dateien};

    %\node(languagemodel)[greyBlock, right= of generator]{Sprachenmodell\\\emph{Abstrakter Syntaxbaum}};
    % Lines  
    \path[arrow, ->] (abstractDescription) -- (wadlAnalysis);
    \path[arrow, ->] (abstractDescription) -- (xsdAnalysis);
    \path[arrow, ->] (wadlAnalysis) -- (restModel);
    \path[arrow, ->] (xsdAnalysis) -- (schemaModel);
    \path[arrow, ->] (restModel) -- (modelCombine);
    \path[arrow, ->] (schemaModel) -- (modelCombine);
    \path[arrow, ->] (modelCombine) -- (applicationModel);
    \path[arrow, ->] (applicationModel) -- (generator);
    \path[arrow, ->] (languageFactory) -- (generator);
    \path[arrow, ->] (generator) -- (languageModel);
    \path[arrow, ->] (languageModel) -- (filePrinter);
    \path[arrow, ->] (filePrinter) -- (library);

    %\path[arrow, ->] (infrastructurecode) -- (generator);
\end{tikzpicture}

    } 
    \caption{Ablaufdiagram des Generators}
    \label{fig:generation_process}
\end{sidewaysfigure}
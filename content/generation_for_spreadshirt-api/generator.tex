\section{Generatorsystem}
\label{sec:generatorsystem}

Ablauf des Generators und Diagramm

\subsection{Language Visitor}
\label{sec:language_visitor}

Kapitel gehört eher zu Generator

% Sprachschnittstelle!
\subsection{Language Factory}
\label{sec:language_factory}

Um eine Zielsprachenunabhänhigkeit zu erreichen, wird dem Generator bei der Erzeugung eine \enquote{Language Factory} übergeben. Der Generator erzeugt Sprachelemente nur über diese Factory. Ein Aufruf einer Factorymethode gibt ein Element der vom Typ der Sprache zurück, der Generator kennt aber nur den Interface-Typ. Für ihn ist die konkrete Implementierung somit transparent.

\subsection{Datenklassen}
\label{sec:dataclasses}

Zielsprachenabhängige Repräsentation

\subsection{Ressourcenklassen}
\label{sec:ressourceclasses}

Nur Zielsprachenabhängig


\subsection{Serialisierer}
\label{sec:serialiser}

Transportorientierte Repräsentation

\subsection{Deserialisierer}
\label{sec:deserialiser}


\subsection{Ausgabemodul}
\label{sec:printer}
% Attribut information aus Modell und nicht aus Datenklassen

\section{Generatorsystem}
\label{sec:generatorsystem}

\Cref{fig:generation_process} zeigt ein Ablaufdiagramm des Generators

\begin{sidewaysfigure}
    \centering
    \resizebox{0.9\textwidth}{!}{
        \begin{tikzpicture}[
        node distance=12mm and 8mm,
        every node/.style={font=\scriptsize}
    ]
    % Blocks
    \node(abstractDescription)[greyBlock]{Abstrakte\\Beschreibung\\der Spreadshirt-API};
    \node(dummy1)[dummy, right=of abstractDescription]{};
    \node(wadlAnalysis)[greyBlock, above right=of abstractDescription]{Analyse\\WADL-Datei};
    \node(restModel)[greyBlock, right=of wadlAnalysis]{REST-\\Modell};
    \node(xsdAnalysis)[greyBlock, below right=of abstractDescription]{Analyse\\XSD-Datei};
    \node(schemaModel)[greyBlock, right=of xsdAnalysis]{Schema-\\Modell};
    \node(modelCombine)[greyBlock, right=of dummy1]{Kombinierer};
    \node(applicationModel)[greyBlock, right=of modelCombine]{Applikations-\\Modell};
    \node(generator)[greyBlock, right=of applicationModel]{Generator};
    \node(languageModel)[greyBlock, right=of generator]{Zielsprachen-\\Modell};
    \node(languageFactory)[greyBlock, below=of generator]{Language Factory};
    \node(filePrinter)[greyBlock, right=of languageModel]{File Printer};
    \node(library)[greyBlock, double copy shadow, right=of filePrinter]{Bibliotheks-\\Dateien};

    %\node(languagemodel)[greyBlock, right= of generator]{Sprachenmodell\\\emph{Abstrakter Syntaxbaum}};
    % Lines  
    \path[arrow, ->] (abstractDescription) -- (wadlAnalysis);
    \path[arrow, ->] (abstractDescription) -- (xsdAnalysis);
    \path[arrow, ->] (wadlAnalysis) -- (restModel);
    \path[arrow, ->] (xsdAnalysis) -- (schemaModel);
    \path[arrow, ->] (restModel) -- (modelCombine);
    \path[arrow, ->] (schemaModel) -- (modelCombine);
    \path[arrow, ->] (modelCombine) -- (applicationModel);
    \path[arrow, ->] (applicationModel) -- (generator);
    \path[arrow, ->] (languageFactory) -- (generator);
    \path[arrow, ->] (generator) -- (languageModel);
    \path[arrow, ->] (languageModel) -- (filePrinter);
    \path[arrow, ->] (filePrinter) -- (library);

    %\path[arrow, ->] (infrastructurecode) -- (generator);
\end{tikzpicture}

    } 
    \caption{Ablaufdiagram des Generators}
    \label{fig:generation_process}
\end{sidewaysfigure}

% Sprachschnittstelle!
\subsection{Language Factory}
\label{sec:language_factory}

Das \emph{Factory}-Pattern behandelt das Problem Familien von Objekten erzeugen zu wollen ohne die konkreten Klassen zu spezifizieren, sondern nur Interfaces festzulegen \cite[][S. 26]{patternsKompakt}.
Um eine Zielsprachenunabhängigkeit zu erreichen, wird aus diesem Grund dem Generator bei der Erzeugung eine Factory übergeben die das Interface \emph{Language Factory} des Sprachmodells implementiert. 

Der Generator erzeugt Sprachelemente nur über diese Factory. Ein Aufruf einer Factorymethode gibt ein Sprachelement der Zielsprache zurück, der Generator kennt aber nur den Interface-Typ. Für ihn ist die konkrete Implementierung somit transparent.

\subsection{Ausgabemodul}
\label{sec:printer}

\subsubsection{Language Visitor}
\label{sec:language_visitor}

Kapitel gehört eher zu Generator
% Attribut information aus Modell und nicht aus Datenklassen

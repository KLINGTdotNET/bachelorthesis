\section{Ausblick}
\label{sec:prospect}

% todo: implementierung anderer Sprachmodelle, Schnittstelle zu Generator

Der in dieser Arbeit dokumentierte Codegenerator bietet mehrere Ansatzpunkte für Erweiterungen oder Verbesserungen:

\begin{compactitem}
    \item Generierung eines \emph{Fluent-Interface} Pattern. Dieses Entwurfsmuster wurde von \citeauthor{fowler2010domain} in \cite{fowler2010domain} beschrieben und basiert auf der Technik des \enquote{method-chaining}, also der Hintereinanderausführung von Methoden, wobei jede Methode mit dem Resultat der vorangegangen arbeitet.
    \item Die Erzeugung von Parameterobjekten welche die Parametersignatur der jeweiligen Methode repräsentieren, wird verhindert das der Nutzer unerlaubte Parameter an eine Methode übergibt. Dies ist derzeit möglich, da vom Nutzer beliebiger Inhalt die Arrays eingetragen werden kann, die zur Übermittlung der Methodenparameter dienen.
    \item Implementieren weiterer Sprachenmodelle, bspw. zur Generierung einer Java-Bibliothek. 
    \item Erzeugen von Tests durch den Generator um die generierte Bibliothek automatisch prüfen zu können.
\end{compactitem}
\chapter{Schlussbetrachtung}
\label{chap:summary}

\chapterQuote{The most important property of a program is whether it accomplishes the intention of its user.}{\citeauthor{hoareAxiomatic}}{\citeyear{hoareAxiomatic}}{\cite[][S. 4]{hoareAxiomatic}}

Das Ziel dieser Arbeit war die Erstellung eines Codegenerators mit flexibler Zielsprache der aus einer abstrakten Beschreibung eines \gls{RESTful} Web Service eine Client-Bibliothek erzeugen sollte. Dies geschah am Beispiel der Spreadshirt-\gls{API} mit \textsc{Php} als Zielsprache der Bibliothek. 
Während der Bearbeitung der Aufgabenstellung stellten sich die folgenden vier Schwerpunkte heraus:
\begin{compactitem}
    \item Erstellung von Datenmodellen für die abstrakte Web Service Beschreibung
    \item Überführung der abstrakten Web Service Beschreibung in die Datenmodelle
    \item Analysieren der Konzepte von objektorientierten Programmiersprachen zur Erzeugung eines Sprachenmodells
    \item Design der Client-Bibliothek auf einfache Nutzung optimieren (beispielweise durch Integration der \gls{API}-Autorisierung)
\end{compactitem}

Der im Rahmen dieser \thesisDesignator{} aus den gewonnen Erkenntnissen entwickelte Codegenerator erzeugt aus der abstrakten Beschreibung der Spreadshirt-\gls{API} eine Client-Bibliothek, die derzeit, wenn auch in eingeschränktem Umfang, nutzbar ist. Da der Generator zielsprachenunabhängig arbeitet, kann er durch Implementierung eines Sprachenmodells eine Client-Bibliothek in einer neuen objektorientierten Zielsprache erzeugen. Durch die Trennung von Syntax und Semantik im Sprachenmodell, ließen sich mit geringem Aufwand Konventionen für den \enquote{code style} der generierten Bibliothek festlgegen. Dies resultiert in gut lesbaren Quellcode der erzeugten Klassen.

Die Aufgabenstellung war umfangreich und fordernd, aber dennoch lösbar. Die Erstellung des Sprachenmodells bedurfte der Auseinandersetzung mit grundlegenden Programmiersprachkonzepten und den Grundlagen des Compilerbaus.

%Im Verlauf dieser \thesisDesignator{} konnten die allgemeinen Konzepte von Web Services, Dokumentbeschreibungssprachen und Codegeneratoren erläutert werden. Aus den beschriebenen Grundlagen konnten Datenmodelle erstellt werden, welche als Ein- und Ausgabemodell des Generators dienen. Das Eingabedatenmodell enthielt die Spreadshirt-\gls{API} Beschreibung und das Ausgabemodell abstrahierte die zu generierende Zielsprache. Im weiteren wurde auch der Ablauf der Codegenerierung besprochen und in einem Diagramm veranschaulicht (\cref{fig:generation_process}). Eine Evaluierung der erzeugten Bibliothek gegenüber den Anforderungen fand ebenfalls statt.

%Im Verlauf dieser \thesisDesignator{} konnte das Zusammenwirken von Datenmodellen und einem 
%Im Verlauf dieser \thesisDesignator{} konnte gezeigt werden wie mit Hilfe der Grundlagen über Web Services, Dokumentbeschreibungs- und objektorientierte Programmiersprachen, darauf basierende Datenmodelle erstellt wurden. 
%Anhand dieser \thesisDesignator{} konnte die Vorgehensweise gezeigt werden ...
%Erstellung einer Client-Bibliothek aus der abstrakten Beschreibung eines Web Services.
%Der entwickelte Codegenerator zur Erzeugung einer Client-Bibliothek für die Spreadshirt-\gls{API} \ldots\\
%Vorteile die sich daraus ergeben\ldots\\

%Schwierigkeiten und Probleme bei der Erstellung\ldots
%    In neue Programmiersprache einarbeiten
%    Überführen der Beschreibung in ein geeignetes Modell
%    Komplexität des Sprachenmodells (Konzepte der Programmiersprachen, Gemeinsamkeiten herausarbeiten)
%    Erstellung des Generators aufwendig 

\section{Ausblick}
\label{sec:prospect}

% todo: implementierung anderer Sprachmodelle, Schnittstelle zu Generator

Der in dieser Arbeit dokumentierte Codegenerator bietet mehrere Ansatzpunkte für Erweiterungen oder Verbesserungen:

\begin{compactitem}
    \item Generierung eines \emph{Fluent-Interface} Pattern. Dieses Entwurfsmuster wurde von \citeauthor{fowler2010domain} in \cite{fowler2010domain} beschrieben und basiert auf der Technik des \enquote{method-chaining}, also der Hintereinanderausführung von Methoden, wobei jede Methode mit dem Resultat der vorangegangen arbeitet.
    \item Die Erzeugung von Parameterobjekten welche die Parametersignatur der jeweiligen Methode repräsentieren, wird verhindert das der Nutzer unerlaubte Parameter an eine Methode übergibt. Dies ist derzeit möglich, da vom Nutzer beliebiger Inhalt die Arrays eingetragen werden kann, die zur Übermittlung der Methodenparameter dienen.
    \item Implementieren weiterer Sprachenmodelle, bspw. zur Generierung einer Java-Bibliothek. 
    \item Erzeugen von Tests durch den Generator um die generierte Bibliothek automatisch prüfen zu können.
\end{compactitem}
\label{sec:prospect}

%\section{Fazit}

%\label{sec:conclusion}

\chapter{Schlussbetrachtung}
\label{chap:summary}

\chapterQuote{Der Zauber steckt immer im Detail.}{\citeauthor{theodorFontane}}{\citeyear{theodorFontane}}{\cite{theodorFontane}}

Im Verlauf dieser \thesisDesignator{} konnten die allgemeinen Konzepte von Web Services, Dokumentbeschreibungssprachen und Codegeneratoren erläutert werden. Aus den beschriebenen Grundlagen konnten Datenmodelle erstellt werden, welche als Ein- und Ausgabemodell des Generators dienen. Das Eingabedatenmodell enthielt die Spreadshirt-\gls{API} Beschreibung und das Ausgabemodell abstrahierte die zu generierende Zielsprache. Im weiteren wurde auch der Ablauf der Codegenerierung besprochen und in einem Diagramm veranschaulicht (\cref{fig:generation_sequence}). Eine Evaluierung der erzeugten Bibliothek gegenüber den Anforderungen fand ebenfalls statt.

%Im Verlauf dieser \thesisDesignator{} konnte das Zusammenwirken von Datenmodellen und einem 
%Im Verlauf dieser \thesisDesignator{} konnte gezeigt werden wie mit Hilfe der Grundlagen über Web Services, Dokumentbeschreibungs- und objektorientierte Programmiersprachen, darauf basierende Datenmodelle erstellt wurden. 
Anhand dieser \thesisDesignator{} konnte die Vorgehensweise gezeigt werden ...
Erstellung einer Client-Bibliothek aus der abstrakten Beschreibung eines Web Services.
Der entwickelte Codegenerator zur Erzeugung einer Client-Bibliothek für die Spreadshirt-\gls{API} \ldots\\
Vorteile die sich daraus ergeben\ldots\\

Probleme bei der Erstellung\ldots
    In neue Programmiersprache einarbeiten
    Überführen der Beschreibung in ein geeignetes Modell
    Komplexität des Sprachenmodells (Konzepte der Programmiersprachen, Gemeinsamkeiten herausarbeiten)
    Erstellung des Generators aufwendig 

\section{Ausblick}
\label{sec:prospect}

% todo: implementierung anderer Sprachmodelle, Schnittstelle zu Generator

Der in dieser Arbeit dokumentierte Codegenerator bietet mehrere Ansatzpunkte für Erweiterungen oder Verbesserungen:

\begin{compactitem}
    \item Generierung eines \emph{Fluent-Interface} Pattern. Dieses Entwurfsmuster wurde von \citeauthor{fowler2010domain} in \cite{fowler2010domain} beschrieben und basiert auf der Technik des \enquote{method-chaining}, also der Hintereinanderausführung von Methoden, wobei jede Methode mit dem Resultat der vorangegangen arbeitet.
    \item Die Erzeugung von Parameterobjekten welche die Parametersignatur der jeweiligen Methode repräsentieren, wird verhindert das der Nutzer unerlaubte Parameter an eine Methode übergibt. Dies ist derzeit möglich, da vom Nutzer beliebiger Inhalt die Arrays eingetragen werden kann, die zur Übermittlung der Methodenparameter dienen.
    \item Implementieren weiterer Sprachenmodelle, bspw. zur Generierung einer Java-Bibliothek. 
    \item Erzeugen von Tests durch den Generator um die generierte Bibliothek automatisch prüfen zu können.
\end{compactitem}
\label{sec:prospect}

%\section{Fazit}

%\label{sec:conclusion}

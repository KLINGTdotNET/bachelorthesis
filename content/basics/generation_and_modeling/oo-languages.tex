\section{Objektorientierte Sprachen}
\label{sec:target_language}

Ziel des Generators ist die Erzeugung von Code in einer Objektorientierten\footnote{nachfolgend nur noch OO} Sprache. Aus diesem Grund werden die elementaren Konzepte solcher Sprachen in diesem Abschnitt näher erläutert, sowie die Besonderheiten der Generatorzielsprache (PHP) beschrieben.

Im Gegensatz zu \emph{Prozeduralen Sprachen}\footnote{Zu den Prozeduralen Sprachen zählt bspw. C und Pascal}, in denen ein Programm eine Liste von Funktionen ist, wird dieses im OO-Programmierparadigma aus der Interaktion von \emph{Objekten} gebildet. 

Ein Objekt kann dabei als eine abgeschlossene Einheit betrachtet werden, die eigene Daten und darauf spezialisierte Methoden besitzt. Durch bestimmte \emph{Sichtbarkeitsregeln} wird festgelegt auf welchen Teil das Objekt nur selber Zugriff hat und welcher Teil von anderen Objekten aufgerufen werden, letztere bilden demnach die Schnittstelle.

% OO gehört zu den Imperativen Sprachen, Prozedurale Programmierung Submenge der Objekt-orientierten Programmierung? 
%Ein Programm in einer OO-Sprache welches nur aus einer Klasse besteht, welche nur öffentliche Methoden besitzt kann direkt auf eine Prozedruale Sprache  abgebildet werden.

\subsection{Konzepte}
\label{sec:concepts_of_object_oriented_languages}

% rekursiv definieren, Objekt enthält methoden und felder, felder und methoden access modifier ...

\begin{description}
    \item[Object]
    \item[Class]
    \item[Method]
    \item[Field]
    \item[Access Modifier]    
\end{description}

\subsection{PHP}
\label{sec:php}

\begin{lstlisting}[language=php]
<?php

class foo {

    private foobar;

    public function __construct() {
        foobar = bar(1);
    } 

    private function bar( /* int */ i) {
        return i++;
    }
}

?>
\end{lstlisting}

Listing, Besonderheiten erwähnen (Syntax)

% todo: Sprachmodell .php basierend auf Datei, da letztendlich auch files generiert werden sollen.

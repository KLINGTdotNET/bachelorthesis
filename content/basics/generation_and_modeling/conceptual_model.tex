\section{Konzeptuelles Modell}
\label{sec:conceptual_model}

\begin{thesisDefinition}[Konzeptuelles Modell]
Ein \emph{Konzeptuelles Modell} beschreibt alle Objekte, Attribute, Rollen und Beziehungen sowie die Beschränkungen einer bestimmten Problemdomäne\footnote{Teil einer Applikation oder Fachbereich, der untersucht werden muss um die Problemstellung zu lösen. \cite{wiki:problemdomain}}.
Bei der Erstellung des Modells werden explizit Design- und Implementierungsentscheidungen außer acht gelassen. 
Das Ziel ist die Bedeutung von Begriffen und Konzepten aus der Domäne zu formalisieren, Beziehungen zwischen den Konzepten finden und Mehrdeutigkeiten bei der Nutzung der Domänentermini zu vermeiden.
\end{thesisDefinition}

Zur Darstellung werden häufig UML- oder ER-Diagramme verwendet, andere Formen sind ebenfalls möglich.
Die Implementierung der Konzepte des Modells kann entweder manuell oder automatisch durch einen Generator erfolgen. Gerade letzteres ist wünschenswert, setzt aber eine große Sorgfalt bei der Erstellung des Modells voraus. Der Generator kann im Gegensatz zum menschlichen Entwickler keine eigenen Schlüße ziehen und muss daher mit den innerhalb des Modells enthaltenen Informationen auskommen.

%Ein Modell bildet die Funktionen und Beziehungen eines Bereiches der Wirklichkeit ab.
%Ein solches Modell, beispielsweise in Form einer \gls{WADL}-Datei zur Beschreibung einer Web-API, dient als Eingabe für einen Generator. Außerdem ist es der Ausgangspunkt in der modell-getriebenen Softwareentwicklung (\gls{MDSD}) oder -architektur (\gls{MDA}).

\begin{figure}[tb]
    \centering
    %\resizebox{\textwidth}{!}{
        \begin{tikzpicture}[
                node distance=12mm and 8mm,
                every node/.style={font=\scriptsize}
            ]
            \node(model)[greyBlock]{Modell};
            \node(generator)[greyBlock, right=of model]{Codegenerator};
            \node(sourcecode)[greyBlock, double copy shadow, right=of generator]{Quellcode};
            \node(templates)[greyBlock, double copy shadow, above=of generator]{Templates};
            \node(infrastructurecode)[greyBlock, double copy shadow, below=of generator]{Infrastrukturcode};     

            \path [arrow, ->] (model) -- (generator);
            \path [arrow, ->] (templates) -- (generator);
            \path [arrow, ->] (infrastructurecode) -- (generator);
            \path [arrow, ->] (generator) -- (sourcecode);
        \end{tikzpicture}   
    %}
    \caption{Simples Generatorsystem}
    \label{fig:generatorsystem}
\end{figure}

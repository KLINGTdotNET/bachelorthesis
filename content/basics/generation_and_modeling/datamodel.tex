\section{Datenmodell}
\label{sec:datamodel}

% todo: Spezifikation oder domänenspezifisches Modell?
Das Datenmodell enthält die Informationen der Spezifikation und dient als Eingabe für den Generator, es ist somit die \emph{Basis der Codegenerierung}. \citeauthor{rfc3198} definieren den Begriff in \cite{rfc3198} folgenderweise\footnote{eigene Übersetzung}:

\thesisDefinition{Datenmodell}{
    Ein Datenmodell ist im Grunde die Darstellung eines Informationsmodells unter Berücksichtigung einer Menge von Mechanismen für die Darstellung, Organisierung, Speicherung und Bearbeitung von Daten.
    Das Modell besteht aus einer Sammlung von \ldots
    \begin{compactitem}
        \item{Datenstrukturen, wie Listen, Tabellen, Relationen etc.}
        \item{Operationen die auf die Strukturen angewendet werden können, wie Abfrage, Aktualisierung, ...}
        \item{Integritätsbedingungen die gültige Zustände (Menge von Werten) odder Zustandsänderungen (Operationen auf Werten) definieren.}
    \end{compactitem}
}

Bei dieser Definition wird der Begriff \emph{Informationsmodell} genutzt, er beschreibt die Informationen die im Datenmodell abgebildet werden sollen ohne Berücksichtigung softwaretechnischer Aspekte. Das Informationsmodell stellt somit die \enquote{natürlichen Daten} dar.

Bei einem Codegenerator entspricht das Datenmodell der internen Darstellung der Spezifikation. Neben den direkt in der Spezifikation enthaltenen Informationen kann der Generator im Analyseschritt (siehe \cref{sec:generator_tasks}) bspw. Datenabhängigkeiten erkennen und diese zur Optimierung nutzen oder das interne Datenmodell damit anreichern. Das \emph{erkennen von Semantik} im Eingabemodell ist aber nicht auf Datenabhängigkeiten beschränkt sondern kann auf beliebige Beziehungen ausgeweitet werden. 

% todo: Datenmodell aus der Spezifikation erstellen, bei REST Vereinigung der WADL und Schemata in ein Datenmodell. Listing zeigt Schnittpunkte zwischen beiden Modellen. Gemeinsames Modell in Form eines Baumes (AST).
% todo: Erklärung warum AST erklärt wird ...

\subsection{Abstract Syntax Tree (AST)}
\label{sec:ast}

Eine anschauliche Definition enthält \cite[][S. 69]{ahoCompiler} (eigene Übersetzung) :
\thesisDefinition{Abstract Syntax Tree -- \citeauthor{ahoCompiler}}{
Ein Abstrakter Syntaxbaum ist die Darstellung eines Ausdrucks, wo jeder Knoten einen \emph{Operator} und dessen Kindknoten die \emph{Operanden} repräsentieren.
Im Allgemeinen kann für jedes Programmierkonstrukt ein Operator erzeugt werden, dessen semantisch bedeutsamen Komponenten dann als Operanden gehandhabt werden.
}

Etwas kürzer definiert \citeauthor{gruneCompiler} den Begriff in \cite[][S. 9 ff.]{gruneCompiler} (eigene Übersetzung):
\thesisDefinition{Abstract Syntax Tree -- \citeauthor{gruneCompiler}}{
Der abstrakte Syntaxbaum stellt die verschiedenen Teile eines Programmtextes aus Sicht der Grammatik, dar.
}

Er ist das Endprodukt eines Parsingschrittes des Quelltextes, im Gegensatz zum \emph{konkreten Syntaxbaum} (auch \emph{Parse Tree}) enthält der \emph{AST} keine Formatierungsspezifische Syntax (bspw. Klammern). 

\begin{figure}[htb]
    \centering
    %\resizebox{\textwidth}{!}{
        \centering
        \begin{tikzpicture}
            \Tree 
                [ .gcd 
                    [ .= 
                        [ .if 
                            [ .== 
                                [ .b ] 
                                [ .0 ]
                            ]
                            [ .a ]
                        ] 
                        [ .else 
                            [ .gcd
                                [ .b ]
                                [ .\% 
                                    [ .a ]
                                    [ .b ]
                                ]
                            ]
                        ]
                    ]
                ]
        \end{tikzpicture}        
    %}
    \caption{Beispiel AST für den rekursiven euklidischen Algorithmus}
    \label{fig:ast}   
\end{figure}

Ein \emph{AST} bildet auch die Grundstruktur des Datenmodells für den Generator. 

%\subsection{PHP Beispiele}

%\section{Konzeptuelles Modell}
\label{sec:conceptual_model}

\begin{thesisDefinition}[Konzeptuelles Modell]
Ein \emph{Konzeptuelles Modell} beschreibt alle Objekte, Attribute, Rollen und Beziehungen sowie die Beschränkungen einer bestimmten Problemdomäne\footnote{Teil einer Applikation oder Fachbereich, der untersucht werden muss um die Problemstellung zu lösen. \cite{wiki:problemdomain}}.
Bei der Erstellung des Modells werden explizit Design- und Implementierungsentscheidungen außer acht gelassen. 
Das Ziel ist die Bedeutung von Begriffen und Konzepten aus der Domäne zu formalisieren, Beziehungen zwischen den Konzepten finden und Mehrdeutigkeiten bei der Nutzung der Domänentermini zu vermeiden.
\end{thesisDefinition}

Zur Darstellung werden häufig UML- oder ER-Diagramme verwendet, andere Formen sind ebenfalls möglich.
Die Implementierung der Konzepte des Modells kann entweder manuell oder automatisch durch einen Generator erfolgen. Gerade letzteres ist wünschenswert, setzt aber eine große Sorgfalt bei der Erstellung des Modells voraus. Der Generator kann im Gegensatz zum menschlichen Entwickler keine eigenen Schlüße ziehen und muss daher mit den innerhalb des Modells enthaltenen Informationen auskommen.

%Ein Modell bildet die Funktionen und Beziehungen eines Bereiches der Wirklichkeit ab.
%Ein solches Modell, beispielsweise in Form einer \gls{WADL}-Datei zur Beschreibung einer Web-API, dient als Eingabe für einen Generator. Außerdem ist es der Ausgangspunkt in der modell-getriebenen Softwareentwicklung (\gls{MDSD}) oder -architektur (\gls{MDA}).

\begin{figure}[tb]
    \centering
    %\resizebox{\textwidth}{!}{
        \begin{tikzpicture}[
                node distance=12mm and 8mm,
                every node/.style={font=\scriptsize}
            ]
            \node(model)[greyBlock]{Modell};
            \node(generator)[greyBlock, right=of model]{Codegenerator};
            \node(sourcecode)[greyBlock, double copy shadow, right=of generator]{Quellcode};
            \node(templates)[greyBlock, double copy shadow, above=of generator]{Templates};
            \node(infrastructurecode)[greyBlock, double copy shadow, below=of generator]{Infrastrukturcode};     

            \path [arrow, ->] (model) -- (generator);
            \path [arrow, ->] (templates) -- (generator);
            \path [arrow, ->] (infrastructurecode) -- (generator);
            \path [arrow, ->] (generator) -- (sourcecode);
        \end{tikzpicture}   
    %}
    \caption{Simples Generatorsystem}
    \label{fig:generatorsystem}
\end{figure}


%\subsection{Domain Specific Language}
\label{sec:dsl}

\emph{Domain Specific Language} (deutsch: \enquote{Domänenspezifische Sprache}) ist eine Programmiersprache die nur auf eine bestimmte Domäne oder auch Problembereich optimiert ist.
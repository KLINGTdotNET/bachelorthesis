\section{Codegeneratoren}

Der Begriff \enquote{Generator} ist sehr allgemein und wird für verschiedene Technologien verwendet, wie \emph{Compiler}, \emph{Präprozessoren}, \emph{Metafunktionen} (Template-Metaprogramming in C++), \emph{Codetransformatoren} und natürlich \emph{Codegeneratoren}. 
\emph{Generator} und \emph{Codegenerator} werden in diesem Kapitel synonym verwendet.

\thesisDefinition{Codegenerator}{
    Ein \emph{Codegenerator} ist ein Programm, welches aus einer höhersprachigen Spezifikation (m. a. W.: auf einem höheren Abstraktionslevel als die zur Implementierung verwendete Programmiersprache), einer Software oder eines Teilaspektes die Implementierung erzeugt (nach \cite{czarnecki2000generative}).
}

Folglich ist der Generator die Schnittstelle zwischen dem \emph{Modell-} und \emph{Implementationsraum}. Der Modellraum beinhaltet das \emph{domänenspezifische Modell}. Dieses Modell wird durch die höhersprachige Spezifikation in einer \emph{Systemspezifikationssprache} beschrieben. In Bezug auf einen \gls{RESTful} Webservice ist beispielsweise \gls{WADL} inklusive seiner verwendeten Schemata die Spezifikationssprache und eine \gls{WADL}-Datei mit den konkreten Spezifikationen demzufolge das domänenspezifische Modell.

Der Informationsgehalt der Spezifikation ist ausschlaggebend für den Grad der zu erreichenden Automatisierung.

\subsection{Aufgaben eines Generators}
\label{sec:generator_tasks}

\begin{compactenum}
    \item[{\footnotesize (optional)}] Analyse der Spezifikation
    \item Validieren der Spezifikation
    \item Spezifikation durch Vorgabewerte vervollständigen
    \item Optimierungen vornehmen
    \item Implementierung erzeugen
\end{compactenum}

Je nach Form der Spezifikation muss diese durch einen Analyse-Schritt (\emph{parsing}) in die interne Darstellung des Generators überführt werden, beispielsweise bei einem Compiler in einen \emph{Abstrakten Syntaxbaum} (\cref{sec:ast}).

\subsection{Vorteile für den Entwickler}
\label{sec:advantages_for_the_developer}

Bei der Nutzung eines Codegenerierungsansatzes ergeben sich nach \cite[][S. 15]{herrington2003code} folgende Vorteile für den Entwickler:

\begin{description}
    \item[Qualität]
        Bugfixes und Verbesserungen werden durch das Generatorsystem in die gesamte Codebasis propagiert.
    \item[Konsistenz]
        Durch ein vorgegebenes Schema für die Schnittstellen- und Variablenbezeichner wird eine hohe Einheitlichkeit erreicht.
    \item[zentrale Wissensbasis]
        Das domänenspezifische Wissen wird in dem Meta- oder auch domänenspezifischen Modell gebündelt, das dem Generator als Eingabe dient. Änderungen am Modell werden durch den Generator in die gesamte Codebasis eingepflegt.
    \item[signifikantere Designentscheidungen]
        Aufgrund des verringerten Implementierungsaufwandes kann der Entwickler mehr Zeit für das Design seiner Architektur, \gls{API} etc. verwenden. Designfehlentscheidungen können durch Änderungen am Modell und nachfolgender Regenierung korrigiert werden und bedürfen somit keiner manuellen Korrektur aller generierten Klassen.
\end{description}

Die Erstellung eines Generatorsystems geht in der Regel mit einem nicht unerheblichen Aufwand einher. Dieser sollte in Relation zum Umfang des zu erzeugenden Codes gesehen werden.
Ist der Umfang des Erzeugnisses zu gering, kann der Aufwand zur Entwicklung einer Generatorlösung kontraproduktiv sein.

\subsection{Generatorformen}
\label{sec:generator_models}

Die folgende Tabelle klassifiziert einige Generatorformen nach der Menge des erzeugten Codes (\enquote{Tier}, zu deutsch: Stufe):
\begin{table}[htb]
    \begin{longtable}[c]{l l l}
        \toprule
        \rowcolor{lightgray}
        \textbf{teilweise}   & \textbf{vollständig}     & \textbf{mehrfach}\\
        \midrule
        Inline-Code Expander    & Tier-Generator    & n-Tier Generator\\
        Mixed-Code Generator & &\\
        Partial-Class Generator & & \\
        \bottomrule
        \caption{Generatoren Klassifikation nach Generierungsmenge}
        \label{tab:generatorclassification}
    \end{longtable} 
\end{table}

\citeauthor{herrington2003code} beschreibt in \cite[Kapitel 4][]{herrington2003code} die Formen aus \Cref{tab:generatorclassification} so:
\begin{description}
    \item[Inline-Code Expander]
        Ein Inline-Code Expander nimmt Quellcode als Eingabe und ersetzt spezielle Mark-Up-Sequenzen durch seine Ausgabe. Die Änderungen werden hierbei nicht in die Quelldatei übernommen, sondern meist direkt zu dem Compiler oder Interpreter weitergeleitet.
    \item[Mixed-Code Generator]
        Der Mixed-Code Generator arbeitet grundsätzlich wie der Inline-Code-Expander, seine Änderungen werden aber in die Quelle zurückgeschrieben.
    \item[Partial-Class Generator]
        Partial-Class Generatoren erzeugen aus einer abstrakten Beschreibung und Templates einen Satz von Klassen, diese bilden aber nicht das vollständige Programm, sondern werden durch manuell erzeugten Code vervollständigt.
    \item[Tier-Generator]
        Die Arbeitsweise des Stufen- oder Tier-Generators entspricht der des Partial-Class Generators, mit der Ausnahme, dass ein vollständiger Code erzeugt wird, der keiner Nacharbeit bedarf.
    \item[$n$-Tier Generator] 
        Ein $n$-Tier Generator erzeugt neben dem eigentlichen Quellcode noch beliebige andere Informationen, beispielsweise eine Dokumentation oder Testfälle.
\end{description}

Die Entwicklung einer \enquote{Full-Domain Language} stellt die oberste Stufe der Generatorformen dar. Eine solche Sprache ist Turing-vollständig und speziell auf die Problemdomäne ausgerichtet.

\subsection{Optimierung durch den Generator}

Die Effektivität von Optimierungen steigt mit dem Abstraktionslevel. Deshalb ist es ratsam diese vom Generator durchführen zu lassen. Im Gegensatz zum Compiler, der viele dieser Optimierungen auch selbst durchführt, besitzt der Generator domänenspezifisches \enquote{Wissen} und kann teilweise ohne Abhängigkeiten der Zielsprache optimieren (\emph{domain-specific optimization}).

% \citeauthor{czarnecki2000generative} beschreiben in \cite{czarnecki2000generative} Optimierungen die vom Generator durchgeführt werden können, einen um \emph{Parallelisierung} erweiterten Auszug bietet die nachfolgende Liste:

% \begin{description}[style=nextline]
% \item[Inlining]
%     Ein Symbol durch seine Definition ersetzen oder anstelle eines Funktionsaufrufes, die Deklaration der Funktion selbst einfügen.
% \item[Constant folding]
%     Auswertung von Ausdrücken deren Operanden während der \emph{compile time} bekannt sind.
% \item[Data caching]
%     Anstatt mehrfach denselben Ausdruck auszuwerten, das Ergebnis einmal berechnen und darauf an anderer Stelle referenzieren.
% \item[Loop fusion]
%     Zusammenführen von Schleifen, die über den gleichen Bereich iterieren und deren Schleifenkörper unabhängig voneinander ist.
% \item[Loop unrolling]
%     Die Schleife durch $n$-mal deren Inhalt ersetzen, wobei $n$ die Anzahl der Iterationen ist.
% \item[Code motion]
%     Invariante\footnote{unveränderliche} Codebereiche aus dem Schleifenkörper herausnehmen.
% \item[Dead-code elimination]
%     Entfernen von ungenutzten Variablen und unerreichbaren Codebereichen.
% \item[Partial evaluation/Specialisation]
%     \emph{Partial evaluation} oder auch \emph{Specialisation} meinen das Erzeugen von spezialisierten Funktionen. Diese implementieren statische Eingaben mit dem Ziel einer kleineren Funktionssignatur.
% \item[Parallelization]
%     Analyse der Datenabhängigkeit durch den Generator und eventuelles parallelisieren voneinander unabhängiger Bereiche.
% \end{description}

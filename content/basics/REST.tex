\section{RESTful Web Service}

\emph{Representational State Transfer} (deutsch: \enquote{Gegenständlicher Zustandstransfer}) ist ein Softwarearchitekturstil für Webanwendungen, welcher von Roy Fielding\footnote{Roy Thomas Fielding, geboren 1965, ist einer der Hauptautoren der HTTP-Spezifikation} in seiner Dissertation aus dem Jahre 2000  beschrieben wurde \cite[Kapitel 5][95 ff.]{fieldingDissertation}. 

Als \gls{RESTful} bezeichnet man dabei eine Webanwendung die den Prinzipien von \gls{REST} entspricht. 

Die fünf Grundlegenden REST-Prinzipien, nach \cite[11 ff.]{Tilkov09}:
\begin{compactitem}
	\item Ressourcen mit eindeutiger Indentifikation
	\item Verknüpfungen / Hypermedia
	\item Standartmethoden\footnote{GET, PUT, POST, DELETE bei Nutzung von HTTP}
	\item Unterschiedliche Repräsentationen
	\item Statuslose Kommunikation
\end{compactitem}

\subsection{Eindeutige Identifikation}
\label{sec:unambigiousidentification}

Um eine \emph{eindeutige Identifikation} zu erreichen, wird jeder Ressource eine \gls{URI} vergeben. Dadurch ist es möglich zu jeder verfügbaren Ressource einen Link zu setzen. 
Nachfolgend eine Beispiel-\gls{URI}, um den Artikel 42 aus dem Warenkorb 84 anzusprechen:
\[
	\underbrace{http://api.spreadshirt.net/api/v1/}_{Basis-URL}\underbrace{\overbrace{baskets/84}^{Warenkorb}\overbrace{/item/42}^{Artikel}}_{Ressource}
	\label{RESTexampleURL}
\]

\subsection{Hypermedia}

Innerhalb einer Ressource kann auf weitere verlinkt werden (\emph{Hypermedia}), als Nebeneffekt der eindeutigen Identifikation duch \glspl{URI} sind diese auch außerhalb des Kontextes der aktuellen Anwendung gültig. Das Folgen eines Links entspricht dabei einer Zustandsänderung innerhalb der Anwendung.
Die vorhandenen Verknüpfungen legen fest welche Zustandsübergänge  erlaubt sind.

\subsection{Standartmethoden}

Durch die Nutzung von \emph{Standartmethoden} ist abgesichert das die Anwendung mit den Ressourcen arbeiten kann, vorausgesetzt sie unterstützt diese. \gls{REST} ist nicht auf HTTP beschränkt, praktisch alle REST-APIs nutzen aber dieses Protokoll. HTTP umfasst dabei folgende Methoden\footnote{Kapitel 9 des HTTP 1.1 RFC2616 beschreibt diese inklusive \emph{TRACE} und \emph{CONNECT} umfassend \cite{HTTP11RFC}}:
\begin{compactitem}
    \item GET
    \item PUT
    \item POST
    \item DELETE
    \item HEAD
    \item OPTIONS
\end{compactitem}

Alle bis auf \emph{POST} und \emph{OPTIONS} sind \emph{idempotent} (\cite{HTTP11RFC} Kapitel 9), d.h. eine hintereinander Ausführung der Methode führt zu demselben Ergebnis wie ein einzelner Aufruf.

\subsection{Repräsentationen von Ressourcen}
\label{sec:representationofresources}

Die Repräsentation sollte unabhängig von der Ressource sein, um die Darstellung gegebenenfalls für den Client anzupassen. Per \emph{Query-Parameter} oder als Information im HTTP-Header kann die Clientanwendung nun das gewünschte Format angeben und bekommt vom Server die entsprechend formatierte Antwort. Der Client kann anhand des \emph{Content-Type} Feldes das Format der Antwort überprüfen, für \gls{JSON} lautet dies bspw. \texttt{application/json}.

\subsection{Statuslose Kommunikation}

Es soll kein Sitzungsstatus (\emph{session-state}) vom Server gespeichert werden, d.h. jede Anfrage des Client muss alle Informationen enthalten, die nötig sind um diese serverseitig verarbeiten zu können. Der Sitzungstatus wird dabei vollständig vom Client gehalten. Diese Restriktion führt zu einigen Vorteilen:
\begin{compactitem}
	\item Verringerung der Kopplung zwischen Client und Server
	\item zwei aufeinanderfolgende Anfragen können von unterschiedlichen Serverinstanzen beantwortet werden
	\item[$\hookrightarrow$] verbesserte Skalierbarkeit
\end{compactitem}
Diese Vorteile werden mit erhöhter Netzwerklast erkauft, da die Statusinformationen bei jeder Anfrage mitgesendet werden müssen.

% Diesen Part über die Grundprinzipien verschieben
\subsection{Elemente von REST}

\subsubsection{Ressource}

Eine Ressource stellt die wichtigste Abstraktion von Information im REST-Modell dar. Fielding definiert eine \emph{ressource} wie folgt:

\thesisquote{Any information that can be named can be a resource: a document or image, \ldots. A resource is a conceptual mapping
to a set of entities, not the entity that corresponds to the mapping at any particular point in
time.}

Eine Ressource kann somit alle Konzepte abbilden die sich über einen Bezeichner referenzieren lassen. Dies können konkrete Dokumente, aber auch Dienste oder sogar Sammlungen von Ressourcen sein.
Außerdem identifiziert ein Ressourcenbezeichner, meist eine \emph{URI} (siehe \cref{sec:unambigiousidentification}), immer dieselbe Ressource, nicht aber deren Wert oder Zustand.

\subsubsection{Repräsentation}

Eine Repräsentation (\emph{representation}) stellt den aktuellen oder den gewünschten Zustand einer Ressource dar. Das Format der Repräsentation ist dabei unabhängig von dem der Ressource, siehe \cref{sec:representationofresources}.
Aktionen mit Komponenten einer REST-API werden durch den Austausch von solchen Repräsentationen durchgeführt.
Im Allgemeinen wird unter einer Repräsentation nur eine Folge von Bytes verstanden, inklusive \emph{Metainformationen} welche den Inhalt der Bytefolge klassifiziert, sowie \emph{Kontrolldaten} die die gewünschte Aktion oder die Bedeutung der Anfrage beschreiben.

\begin{minipage}{\textwidth}
    \begin{lstlisting}[
        language=XML,
        caption=Beispiel zu Metainformationen für REST-Repräsentation aus WADL-Datei der Spreadshirt-API,
        label=lst:metainformationREST
        ]
    ...
    <response>
        <representation xmlns:sns="http://api.spreadshirt.net" //@{\ding{202}\,}@//element="sns:productTypes" status="200" //@{\ding{203\,}@//mediaType="application/xml">
            <doc title="Success"/>
        </representation>
        ...
    </response>
    \end{lstlisting}
\end{minipage}

Ein Beispiel für eine solche Angabe von Metainformationen ist in \Cref{lst:metainformationREST} zu finden, \ding{202} zeigt dies in Form einer \emph{Typangabe} und eines \emph{mediaType}-Attributes \ding{203}.

Kontrolldaten sind meist HTTP-Header-Felder\footnote{für die HTTP-Header-Definitionen siehe \url{http://www.w3.org/Protocols/rfc2616/rfc2616-sec14.html\#sec14.43}}, bspw. um das Cachingverhalten zu ändern.

\subsubsection{Konnektoren}
\chapter{Datenmodell}

\begin{thesisDefinition}[Datenmodell]
Das Datenmodell, oder auch \enquote{Datenstruktur} im Kontext von Programmiersprachen genannt, beschreibt die Beziehungen, Einschränkungen, Semantik und Struktur der darin enthaltenen Daten \cite{wiki:datamodel}.
\end{thesisDefinition}

Im Allgemeinen enthält das Modell die Daten von den Geschäftsprozessen gebraucht und erzeugt werden und erfasst explizit deren Struktur. Modelle werden meist in einer speziellen (grafischen) Notation beschrieben (bspw. ER-Diagramm). 

Einteilung der Datenmodellschemas nach ANSI: %ToDo: Bibtex entry
\begin{description}
    \item[Konzeptuelles Schema]
        beschreibt die Semantiken der Domäne und legt den Gültigkeitsbereich des Datenmodells fest.
    \item[Logisches Schema]
        dient der Strukturbeschreibung des Modells. Wird oft durch Verfeinerung aus dem Konzeptuellen Schema entwickelt.
    \item[Physisches Schema]
        definiert die Form der Speicherung der Daten auf einem Physischen Datenträger.
\end{description}

Das Datenmodell dient als Eingabe für den Generator und ist somit die Basis für die Codegenerierung. Bei der Erstellung ist darauf zu achten das Modell möglichst präzise zu formulieren um Mehrdeutigkeiten bei der Interpretation zu vermeiden. 
Neben den direkt im Modell enthaltenen Informationen kann der Generator auch Abhängigkeiten darin erkennen und das Modell damit semantisch anreichern. Beispielweise die Beziehungen zwischen Datentypen die aus der \gls{XSD} extrahiert wurden \ldots

% konkretes Modell

Das konkrete Modell für die Spreadshirt-API, setzt sich aus deren Beschreibung im \gls{WADL} und \gls{XSD} zusammen. Die Schnittpunkte beider Beschreibungen finden sich in der Angabe der Methodenparameter- und Rückgabewerte der \gls{WADL}-Datei. 

\begin{lstlisting}[
    language=XML,
    label=lst:wadlintersection,
    caption=HTTP-GET-Methode zur Anzeige von Produkttypen eines Users (gekürzt)
    ]
<resource path="users/{userId}/productTypes">
    ...
    <method name="GET">
        ...
        <request>
            ...
            <param xmlns:xsd="http://www.w3.org/2001/XMLSchema"//@\ding{204}@// name="sessionId" style="query" type="xsd:string" //@\ding{202}@//>
                <doc>The security session id, e.g. 123.</doc>
            </param>
            ...
        </request>
        <response>
            <representation xmlns:sns="http://api.spreadshirt.net"//@\ding{205}@// element="sns:productTypes" //@\ding{203}@// status="200" mediaType="application/xml">
                <doc title="Success"/>
            </representation>
            <fault status="500" mediaType="text/plain">
                <doc title="Internal Server Error"/>
            </fault>
            ...
        </response>
    </method>
</resource>
\end{lstlisting}

\Cref{lst:wadlintersection} zeigt einen Ausschnitt aus der Beschreibung der API-Methode zum Erhalt der \emph{Produkttypen} eines bestimmten \emph{Users}. Punkt \ding{202} und \ding{203} sind Typangaben, wobei ersteres einen Parameter- und letzteres einen Rückgabetyp ist. Die jeweiligen Namensräume der Typen werden unter \ding{204} und \ding{205} angegeben.

%Als Eingabe des Generators dient ein \emph{Datenmodell}, welches sich aus dem Inhalt der Webanwendungsbeschreibung (\cref{sec:wadl}) und dem der Datentypbeschreibungen (\cref{sec:xsd}) zusammensetzt. Neben den direkt aus den jeweiligen Beschreibungsformaten abzulesenden Informationen werden durch den Generator Abhängigkeiten zwischen den Datentypen erkannt und dem Modell hinzugefügt. %Todo: Verweis auf parsing step indem dies geschieht
%Die Struktur des Modells entspricht einem \gls{AST}

\section{Abstract Syntax Tree (AST)}

\begin{thesisDefinition}[Abstract Syntax Tree -- Grune]
Der abstrakte Syntaxbaum (engl. \enquote{Abstract Syntax Tree}) stellt die verschiedenen Teile eines Programmtextes aus Sicht der Grammatik, dar \cite[][S. 9 ff.]{gruneCompiler}.
\end{thesisDefinition}
\begin{thesisDefinition}[Abstract Syntax Tree -- Aho]
Ein Abstrakter Syntaxbaum ist die Darstellung eines Ausdrucks, wo jeder Knoten einen \emph{Operator} und dessen Kindknoten die \emph{Operanden} repräsentieren.
%Jeder Knoten eines \emph{Abstrakten Syntax Baumes} für einen Ausdruck, repräsentiert einen \emph{Operator} und dessen Kindknoten stellen die  \emph{Operanden} dar. 
Im Allgemeinen kann für jedes Programmierkonstrukt ein Operator erzeugt werden, dessen semantisch bedeutsamen Komponenten dann als Operanden gehandhabt werden (nach \cite[][S. 69]{ahoCompiler}).
\end{thesisDefinition}
Er ist das Endprodukt eines Parsingschrittes des Quelltextes, im Gegensatz zum \emph{konkreten Syntaxbaum} (auch \emph{Parse Tree}) enthält der \emph{AST} keine Formatierungsspezifische Syntax (bspw. Klammern). 

\begin{figure}[htb]
    \centering
    %\resizebox{\textwidth}{!}{
        \centering
        \begin{tikzpicture}
            \Tree 
                [ .gcd 
                    [ .= 
                        [ .if 
                            [ .== 
                                [ .b ] 
                                [ .0 ]
                            ]
                            [ .a ]
                        ] 
                        [ .else 
                            [ .gcd
                                [ .b ]
                                [ .\% 
                                    [ .a ]
                                    [ .b ]
                                ]
                            ]
                        ]
                    ]
                ]
        \end{tikzpicture}        
    %}
    \caption{Beispiel AST für den rekursiven euklidischen Algorithmus}
    \label{fig:ast}   
\end{figure}

\section{PHP Beispiele}

\begin{lstlisting}[
    language=PHP,
    caption=Abfrage der Designs eines Users per HTTP-GET
    ]
<?php

?>
\end{lstlisting}

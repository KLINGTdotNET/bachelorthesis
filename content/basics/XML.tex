\subsection{XML}

\begin{thesisDefinition}[XML]
Die \emph{Extensible Markup Language}, kurz \gls{XML}, ist eine Auszeichnungssprache (\enquote{Markup Language}) die eine Menge von Regeln beschreibt um Dokumente in einem mensch- und maschinen lesbaren Format zu kodieren \cite{XML10Specification}.
\end{thesisDefinition}
Obwohl das Design von XML auf Dokumente ausgerichtet ist, wird es häufig für die Darstellung von beliebigen Daten benutzt \cite{wiki:xml}, z.B. um diese für die Übertragung zu serialisieren.

\begin{lstlisting}[
        language=XML, 
        caption=Die gekürzte Antwort der API-Ressource \emph{users/{userid}/designs/{designID}} als Beispiel für eine XML-Datei
    ][htb]
<?xml version="1.0" encoding="UTF-8" standalone="yes"?> //@\ding{202}@//
<design xmlns:xlink="http://www.w3.org/1999/xlink" 
        xmlns="http://api.spreadshirt.net" 
        ...>//@\ding{203}@//
    <name>tape_recorder</name>
    ...
    <size unit="px">
        <width>3340.0</width>
        <height>3243.0</height>
    </size>
    <colors/>//@\ding{204}@//
    ...
    <created>
        2013-03-30T12:37:54Z //@\ding{205}@//
    </created>
    <modified>2013-04-02T11:13:02Z</modified>
</design>//@\ding{206}@//
\end{lstlisting}

Eine valide XML-Datei beginnt mit der \emph{XML-Deklaration} \ding{202}, diese enthält Angaben über die verwendete XML-Spezifikation und die Kodierung der Datei. 
Im Gegensatz zu gewöhnlichen Tags, wird dieses mit \texttt{<?} und mit \texttt{?>} beendet. 
Danach folgen beliebig viele baumartig geschachtelte \emph{Elemente} mit einem Wurzelelement \ding{203}. Die Elemente können Attribute enthalten und werden, wenn sie kein leeres Element sind \ding{204}, von einem schließenden Tag in der gleichen Stufe abgeschlossen \ding{206}. Nicht leere Zeichenketten als Kindelement sind ebenfalls erlaubt \ding{205}.

Mit Hilfe von \emph{Schemabeschreibungssprachen} (siehe \cref{sec:xmlschema}) kann der Inhalt und die Struktur eines Dokumentes festgelegt und gegen diese validiert werden. Der Begriff \emph{XML Schema} ist mehrdeutig und wird oft auch für eine konkrete Beschreibungssprache, die \enquote{XML Schema Definition}, kurz \gls{XSD}, verwendet.

\chapter{Generierung und Modellierung}
\label{chap:generation_and_modeling}

\chapterQuote{Any problem in computer science can be solved with another level of indirection.}{David Wheeler}{1993}{Turing Award Lecture. February 17, 1993.}

% Todo: Kapiteleinleitung für Generatoren und Modellierung
Im folgenden Kapitel werden grundlegende Begriffe im Zusammenhang mit Codegeneratoren und Datenmodellen definiert, zusätzlich wird eine Übersicht über gebräuchliche Generatorformen, dessen Aufgaben und Arten der Optimierung durch den Generator, gegeben. 
\Cref{sec:datamodel} \ldots

%Generative Programming p. 333
%todo: \cite{czarnecki2000generative} in der Kapiteleinleitung erwähnen um sich wiederholende Referenzen zu vermeiden

\begin{thesisDefinition}[Codegenerator]
Ein \emph{Codegenerator} ist ein Programm, welches aus einer höhersprachigen Spezifikation\footnote{m. a. W.: auf einem höheren Abstraktionslevel als die zur Implementierung verwendete Programmiersprache} einer Software oder eines Teilaspektes, die Implementierung erzeugt (nach \cite{czarnecki2000generative}).
\end{thesisDefinition}

% todo: folgende Liste entfernen?
Generatoren widmen sich drei wichtigen Problemen\cite{czarnecki2000generative}:
\begin{description}[style=nextline]
    \item[Relevanz von Systembeschreibungen erhöhen] Eine Systembeschreibung sollte direkt und explizit die Anforderungen bestimmen und mit der Sprache der Problemdomäne formuliert sein.
    \item[Erzeugung einer effizienten Implementierung] Die größte Herausforderung bei der Erstellung eines Generators liegt in der Abbildung von der Spezifikation zur Implementierung, da es meist keine direkte Übereinstimmung zwischen beiden Konzepten gibt.
    \item[\enquote{Library scaling problem}] Nur die durch die Spezifikation benötigten Methoden generieren.
\end{description}

% todo: Satz ändern, quasi nur Übersetzung, Code-Transformatoren nicht erklärt
Der Begriff \enquote{Generator} ist sehr allgemein und wird für verschiedene Technologien verwendet, wie \emph{Compiler}, \emph{Präprozessoren}, \emph{Metafunktionen} (Template-Metaprogramming in C++), \emph{Codetransformatoren} und natürlich \emph{Codegeneratoren}.

\section{Aufgaben eines Generators}
\begin{compactenum}
    \item Validieren der Spezifikation
    \item Spezifikation durch Vorgabewerte vervollständigen
    \item Optimierungen vornehmen
    \item Implementierung erzeugen
\end{compactenum}

Je nach Form der Spezifikation, muss diese durch einen Analyse-Schritt (\emph{parsing}) in die interne Darstellung des Generators überfürht werden, bspw. bei einem Compiler in einen \emph{Abstrakten Syntaxbaum} (siehe \cref{sec:ast}).
Der Informationsgehalt der Spezifikation ist ausschlaggebend für den Grad der zu erreichenden Automatisierung.

\section{Vorteile für den Entwickler}

Die folgende Auflistung basiert ebenfalls auf \cite[][S. 15]{czarnecki2000generative}.

\begin{description}
    \item[Qualität]
        Bugfixes und Verbesserungen werden durch das Generatorsystem in die gesamte Codebasis propagiert.
    \item[Konsistenz]
        Durch ein vorgegebenes Schema für die Schnittstellen- und Variablenbezeichner wird eine hohe Einheitlichkeit erreicht.
    \item[zentrale Wissensbasis]
        % todo: Metamodell erklären
        Das domänenspezifische Wissen wird in dem Metamodell gebündelt, das dem Generator als Eingabe dient. Änderungen am Modell werden durch den Generator in die gesamte Codebasis eingepflegt.
    \item[signifikantere Designentscheidungen]
        Aufgrund des verringerten Implementierungsaufwandes kann der Entwickler mehr Zeit für das Design seiner Architektur , API etc. verwenden. Designfehlentscheidungen können durch Änderungen an den Templates korrigiert werden und bedürfen somit keiner manuellen Korrektur aller generierten Klassen.
\end{description}

Die Erstellung eines Generatorsystems geht mit einem nicht unerheblichen Aufwand einher, dieser sollte in Relation zum Umfang des zu erzeugenden Codes gesehen werden.
Ist der Umfang des Erzeugnisses zu gering, kann der Aufwand zur Entwicklung einer Generatorlösung kontraproduktiv sein.

\section{Generatorformen}

Eine Form der Klassifikation von Codegeneratoren ist nach der Menge des erzeugten Codes:

\begin{table}[htb]
    \begin{longtable}[c]{l l l}
        \toprule
        \rowcolor{lightgray}
        \textbf{teilweise}   & \textbf{vollständig}     & \textbf{mehrfach}\\
        \midrule
        Inline-Code Expander    & Tier-Generator    & n-Tier Generator\\
        Mixed-Code Generator & &\\
        Partial-Class Generator & & \\
        \bottomrule
        \caption{Generatoren Klassifikation nach Generierungsmenge}
        \label{tab:generatorclassification}
    \end{longtable} 
\end{table}

Generatorformen nach \cite{herrington2003code}:
\begin{description}
    \item[Inline-Code Expander]
        Ein Inline-Code Expander nimmt Quellcode als Eingabe und ersetzt spezielle Mark-Up Sequenzen durch seine Ausgabe. Die Änderungen werden hierbei nicht in das Quellfile übernommen sondern meist direkt zu dem Compiler oder Interpreter weitergeleitet.
    \item[Mixed-Code Generator]
        Der Mixed-Code Generator arbeitet grundsätzlich wie der Inline-Code Expander, seine Änderungen werden aber in die Quelle zurückgeschrieben.
    \item[Partial-Class Generator]
        Partial-Class Generatoren erzeugen aus einer abstrakten Beschreibung und Templates einen Satz von Klassen, diese bilden aber nicht das vollständige Programm sondern werden durch manuell erzeugten Code vervollständigt. % todo: manuell erzeugt = handgeschrieben?
    \item[Tier-Generator]
        Die Arbeitsweise des Stufen- oder Tier\footnote{Tier, zu deutsch \enquote{Stufe}}-Generators entspricht der des Partial-Class Generators, mit der Ausnahme das vollständiger Code erzeugt wird, der keiner Nacharbeit bedarf.\footnote{Der im Laufe dieser Arbeit entwickelte Generator entspricht diesem Schema.}
    \item[$n$-Tier Generator] 
        Ein $n$-Tier Generator erzeugt neben dem eigentlichen Quellcode noch beliebige andere Informationen, bspw. eine Dokumentation oder Testfälle.
\end{description}

Die Entwicklung einer \enquote{Full-Domain Language} stellt die oberste Stufe der Generatorformen dar. Eine solche Sprache ist Turing-vollständig und speziell auf die Problemdomäne ausgerichtet.

\section{Optimierung durch den Generator}

Die Effektivität von Optimierungen steigt mit dem Abstraktionslevel, deshalb ist es ratsam diese vom Generator durchführen zu lassen. Im Gegensatz zum Compiler, der viele dieser Optimierungen auch selbst durchführt, besitzt der Generator domänenspezifisches \enquote{Wissen} (\emph{domain-specific optimization}) und kann teilweise ohne Abhängigkeiten der Zielsprache optimieren.

Nachfolgend eine Auflistung möglicher Optimierungen:
% global optimizations?

\begin{description}[style=nextline]
\item[Inlining]
    Ein Symbol durch seine Definition ersetzen oder anstelle eines Funktionsaufrufes, die Deklaration der Funktion selbst einfügen.
\item[Constant folding]
    Auswertung von Ausdrücken deren Operanden während der \emph{compile time} bekannt sind.
\item[Data caching]
    Anstatt mehrfach denselben Ausdruck auszuwerten, das Ergebnis einmal berechnen und darauf an anderer Stelle referenzieren.
\item[Loop fusion]
    Zusammenführen von Schleifen, die über den gleichen Bereich iterieren und deren Schleifenkörper unabhängig voneinander ist.
\item[Loop unrolling]
    Die Schleife durch $n$-mal deren Inhalt ersetzen, wobei $n$ die Anzahl der Iterationen ist.
\item[Code motion]
    Invariante\footnote{unveränderliche} Codebereiche aus dem Schleifenkörper herausnehmen.
\item[Dead-code elimination]
    Entfernen von ungenutzten Variablen und unerreichbaren Codebereichen.
\item[Partial evaluation/Specialisation]
    \emph{Partial evaluation} oder auch \emph{Specialisation} meinen das Erzeugen von spezialisierten Funktionen. Diese implementieren statische Eingaben mit dem Ziel einer kleineren Funktionssignatur.
\item[Parallelization]
    Analyse der Datenabhängigkeit durch den Generator und eventuelles parallelisieren voneinander unabhängiger Bereiche.
\end{description}


\section{Datenmodell}
\label{sec:datamodel}

% todo: Spezifikation oder domänenspezifisches Modell?
Das Datenmodell enthält die Informationen der Spezifikation und dient als Eingabe für den Generator, es ist somit die \emph{Basis der Codegenerierung}. \citeauthor{rfc3198} definieren den Begriff in \cite{rfc3198} folgenderweise\footnote{eigene Übersetzung}:

\thesisDefinition{Datenmodell}{
    Ein Datenmodell ist im Grunde die Darstellung eines Informationsmodells unter Berücksichtigung einer Menge von Mechanismen für die Darstellung, Organisierung, Speicherung und Bearbeitung von Daten.
    Das Modell besteht aus einer Sammlung von \ldots
    \begin{compactitem}
        \item{Datenstrukturen, wie Listen, Tabellen, Relationen etc.}
        \item{Operationen die auf die Strukturen angewendet werden können, wie Abfrage, Aktualisierung, ...}
        \item{Integritätsbedingungen die gültige Zustände (Menge von Werten) odder Zustandsänderungen (Operationen auf Werten) definieren.}
    \end{compactitem}
}

Bei dieser Definition wird der Begriff \emph{Informationsmodell} genutzt, er beschreibt die Informationen die im Datenmodell abgebildet werden sollen ohne Berücksichtigung softwaretechnischer Aspekte. Das Informationsmodell stellt somit die \enquote{natürlichen Daten} dar.

Bei einem Codegenerator entspricht das Datenmodell der internen Darstellung der Spezifikation. Neben den direkt in der Spezifikation enthaltenen Informationen kann der Generator im Analyseschritt (siehe \cref{sec:generator_tasks}) bspw. Datenabhängigkeiten erkennen und diese zur Optimierung nutzen oder das interne Datenmodell damit anreichern. Das \emph{erkennen von Semantik} im Eingabemodell ist aber nicht auf Datenabhängigkeiten beschränkt sondern kann auf beliebige Beziehungen ausgeweitet werden. 

Wie man in \Cref{fig:wadlstructure} erkennen kann, entspricht die Struktur einer WADL-Datei einem Baum. Aus diesem Grund eignet sich eine Baumstruktur für das Datenmodell des Generators besonders gut. Um die Zielsprache (siehe \cref{sec:target_language}) im internen Datenmodell abbilden zu können, wird ein \gls{AST} als Datenstruktur gewählt.

\subsection{Abstract Syntax Tree (AST)}
\label{sec:ast}

Eine anschauliche Definition bietet \cite[][S. 69]{ahoCompiler} (eigene Übersetzung):
\thesisDefinition{Abstract Syntax Tree -- \citeauthor{ahoCompiler}}{
Ein Abstrakter Syntaxbaum ist die Darstellung eines Ausdrucks, wo jeder Knoten einen \emph{Operator} und dessen Kindknoten die \emph{Operanden} repräsentieren.
Im Allgemeinen kann für jedes Programmierkonstrukt ein Operator erzeugt werden, dessen semantisch bedeutsamen Komponenten dann als Operanden gehandhabt werden.
}

%Etwas kürzer definiert \citeauthor{gruneCompiler} den Begriff in \cite[][S. 9 ff.]{gruneCompiler} (eigene Übersetzung):
% \thesisDefinition{Abstract Syntax Tree -- \citeauthor{gruneCompiler}}{ Der abstrakte Syntaxbaum stellt die verschiedenen Teile eines Programmtextes aus Sicht der Grammatik, dar. }

Er ist das Endprodukt eines Parsingschrittes des Quelltextes, im Gegensatz zum \emph{konkreten Syntaxbaum} (auch \emph{Parse Tree}) enthält der \emph{AST} keine Formatierungsspezifische Syntax (bspw. Klammern). 

\begin{figure}[htb]
    \centering
    %\resizebox{\textwidth}{!}{
        \centering
        \begin{tikzpicture}
            \Tree 
                [ .gcd 
                    [ .= 
                        [ .if 
                            [ .== 
                                [ .b ] 
                                [ .0 ]
                            ]
                            [ .a ]
                        ] 
                        [ .else 
                            [ .gcd
                                [ .b ]
                                [ .\% 
                                    [ .a ]
                                    [ .b ]
                                ]
                            ]
                        ]
                    ]
                ]
        \end{tikzpicture}        
    %}
    \caption{Beispiel AST für den rekursiven euklidischen Algorithmus}
    \label{fig:ast}   
\end{figure}

%Ein \emph{AST} bildet auch die Grundstruktur des Datenmodells für den Generator. 


\section{Objektorientierte Sprachen}
\label{sec:target_language}

Ziel des Generators ist die Erzeugung von Code in einer Objektorientierten\footnote{nachfolgend nur noch OO} Sprache. Aus diesem Grund werden die elementaren Konzepte solcher Sprachen in diesem Abschnitt näher erläutert, sowie die Besonderheiten der Generatorzielsprache (PHP) beschrieben.

Im Gegensatz zu \emph{Prozeduralen Sprachen}\footnote{Zu den Prozeduralen Sprachen zählt bspw. C und Pascal}, in denen ein Programm eine Liste von Funktionen ist, wird dieses im OO-Programmierparadigma aus der Interaktion von \emph{Objekten} gebildet. 

Ein Objekt kann dabei als eine abgeschlossene Einheit betrachtet werden, die eigene Daten und darauf spezialisierte Methoden besitzt. Durch bestimmte \emph{Sichtbarkeitsregeln} wird festgelegt auf welchen Teil das Objekt nur selber Zugriff hat und welcher Teil von anderen Objekten aufgerufen werden, letztere bilden demnach die Schnittstelle.

% OO gehört zu den Imperativen Sprachen, Prozedurale Programmierung Submenge der Objekt-orientierten Programmierung? 
%Ein Programm in einer OO-Sprache welches nur aus einer Klasse besteht, welche nur öffentliche Methoden besitzt kann direkt auf eine Prozedruale Sprache  abgebildet werden.

\subsection{Konzepte}
\label{sec:concepts_of_object_oriented_languages}

% rekursiv definieren, Objekt enthält methoden und felder, felder und methoden access modifier ...

\begin{description}
    \item[Object]
    \item[Class]
    \item[Method]
    \item[Field]
    \item[Access Modifier]    
\end{description}

\subsection{PHP}
\label{sec:php}

\begin{lstlisting}[language=php]
<?php

class foo {

    private foobar;

    public function __construct() {
        foobar = bar(1);
    } 

    private function bar( /* int */ i) {
        return i++;
    }
}

?>
\end{lstlisting}

Listing, Besonderheiten erwähnen (Syntax)

% todo: Sprachmodell .php basierend auf Datei, da letztendlich auch files generiert werden sollen.

\subsection{JSON}

\begin{longtable}[htb]{l l}
    \toprule
    \rowcolor{lightgray}
    \textbf{primitiv}   & \textbf{strukturiert}\\
    \midrule
    Zeichenketten       & Objekte\\
    Ganz- und 
    Fließkommazahlen    & Arrays\\
    Booleans            & \\
    null                & \\
    \bottomrule
    \caption{JSON Datentypen}
    \label{tab:jsonDatatypes}
\end{longtable}

\emph{Javascript Object Notation}, kurz \emph{JSON}, ist ein leichtgewichtiges, textbasiertes und sprachunbhängiges Datenaustauschformat. Es ist von \emph{JavaScript} abgeleitet und definiert eine kleine Menge von Formatierungsregeln für die transportable Darstellung (Serialisierung) von strukturierten Daten (nach \cite{JSONRFC}).

Im Gegensatz zu XML ist JSON weit weniger mächtig, es gibt z.B. keine Unterstützung für Namensräume und es wird nur eine geringe Menge an Datentypen unterstützt (siehe \cref{tab:jsonDatatypes}). 
Durch seine einfache Struktur wird aber ein deutlich geringerer \enquote{syntaktischen Overhead} erzeugt.
Mit \printhref{http://tools.ietf.org/id/draft-zyp-json-schema-03.html}{JSON Schema} ist es möglich eine Dokumentstruktur vorzugeben und gegen diese zu validieren. 

%
% wrapping in a minipage prevents the listings block from splitting on pagebreak
%
\begin{minipage}{\textwidth}
    \begin{lstlisting}[language=JavaScript, caption=Minimalbeispiel für eine JSON-Datei]
{
    "key": "value",
    "another_key": 42,
    "an_object": {
        "foo": "bar",
        "some_boolean": false
    },
    "an_array": [
        "surname": "Andreas",
        "lastname": "Linz"
    ],
    "empty": null
}
    \end{lstlisting}
\end{minipage}
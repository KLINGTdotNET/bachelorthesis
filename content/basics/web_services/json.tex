\subsection{JSON}
\label{sec:json}

\begin{thesisDefinition}[JSON]
\emph{Javascript Object Notation}, kurz \emph{JSON}, ist ein leichtgewichtiges, textbasiertes und sprachunbhängiges Datenaustauschformat. Es ist von \emph{JavaScript} abgeleitet und definiert eine kleine Menge von Formatierungsregeln für die transportable Darstellung (Serialisierung) von strukturierten Daten (nach \cite{JSONRFC}).
\end{thesisDefinition}

Im Gegensatz zu XML ist JSON weit weniger mächtig, es gibt z.B. keine Unterstützung für Namensräume und es wird nur eine geringe Menge an Datentypen unterstützt (siehe \cref{tab:jsonDatatypes}). 
Durch seine einfache Struktur wird aber ein deutlich geringerer \enquote{syntaktischer Overhead} erzeugt.
Mit \printhref{http://tools.ietf.org/id/draft-zyp-json-schema-03.html}{JSON Schema} ist es möglich eine Dokumentstruktur vorzugeben und gegen diese zu validieren. 

\begin{table}[tb]
    \begin{longtable}[c]{l l}
        \toprule
        \rowcolor{lightgray}
        \textbf{primitiv}   & \textbf{strukturiert}\\
        \midrule
        Zeichenketten       & Objekte\\
        Ganz- und 
        Fließkommazahlen    & Arrays\\
        Booleans            & \\
        null                & \\
        \bottomrule
        \caption{JSON Datentypen}
        \label{tab:jsonDatatypes}
    \end{longtable}
\end{table}

Objekte werden in JSON von geschweiften- \ding{202}, Arrays hingegen von geschlossenen Klammern begrenzt \ding{203}. Diese dürfen beliebig tief geschachtelt werden und auch Schlüssel-Wert-Paare (\emph{key-value-pairs} \ding{204}) enthalten. \emph{Schlüssel sind immer Zeichenketten}, die Werte dürfen von allen Typen aus \cref{tab:jsonDatatypes} sein.
%
% wrapping in a minipage prevents the listings block from splitting on pagebreak
%
\begin{lstlisting}[
    language=JavaScript,
    caption=Die gekürzte Antwort der API-Ressource \emph{users/{userid}/designs/{designID}} als Beispiel für eine JSON-Datei
]
{
    "name": "tape_recorder", //@\ding{204}@//
    "description": "",
    "user": { //@\ding{202}@//
        "id": "1956580",
        "href": "http://api.spreadshirt.net/api/v1/users/1956580"
    }, //@\ding{202}@//
    "resources": [ //@\ding{203}@//
    ...
        {
            "mediaType": "png",
            "type": "preview",
            "href": "http://image.spreadshirt.net/image-server/v1/designs/15513946"
        }, 
        ...
    ], //@\ding{203}@//
    "created": "30.03.2013 12:37:54",
    ...
}
\end{lstlisting}    

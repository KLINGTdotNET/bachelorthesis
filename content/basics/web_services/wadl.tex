\section{WADL}
\label{sec:wadl}

Die \emph{Web Application Description Language} (kurz \gls{WADL}) ist eine maschinenlesbare Beschreibung einer \textsc{Http}-basierten Webanwendung, einschließlich einer Menge von \textsc{Xml} Schematas \cite{hadleyWADL}.
Die aktuelle Revision ist vom \citedate{WADLcurrent} \cite{WADLcurrent}, im weiteren beziehe ich mich aber auf die in der Spreadshirt-\textsc{Api} verwendeten Version, datiert am 9. November 2006. Die Unterschiede zwischen beiden Revisionen können unter \cite{WADLchanges} nachvollzogen werden.

Die Beschreibung eines Webservices durch \textsc{Wadl} besteht nach \cite{hadleyWADL} im groben aus den folgenden vier Bestandteilen:
\begin{description}
     \item[Set of resources] Analog einer Sitemap, die Übersicht aller verfügbaren Ressourcen.
     \item[Relationships between resources] Die kausale und referentielle Verknüpfung zwischen Ressourcen.
     \item[Methods that can be applied to each resource] Die von der jeweiligen Ressource unterstützten [\textsc{Http}]-Methoden, deren Ein- und Ausgabe, sowie die unterstützten Formate.
     \item[Resource representation formats] Die unterstützten \gls{MIME}-Typen und verwendeten Datenschemas (\cref{sec:xsd}).
 \end{description} 

%\begin{minipage}{\textwidth}
\begin{lstlisting}[language=XML, caption={Beispielaufbau einer \textsc{Wadl}-Datei anhand der Spreadshirt-\textsc{Api} Beschreibung}, label=lst:wadlstructure, name=wadlstructure]
<?xml version="1.0" encoding="UTF-8" standalone="yes"?> //@\ding{202}@//
<application xmlns="http://research.sun.com/wadl/2006/10"> //@\ding{203}@//
    <grammars> //@\ding{204}@//
        <include href="http://api.spreadshirt.net/api/v1/metaData/api.xsd">
            <doc>Catalog XML Schema.</doc>
        </include>
        ...
    </grammars>
    <resources base="http://api.spreadshirt.net/api/v1/"> //@\ding{205}@//
        <resource path="users/{userId}"> //@\ding{206}@//
            <doc>Return user data.</doc>
            <method name="GET"> //@\ding{207}@//
                <doc>...</doc>
                <request> //@\ding{208}@//
                    <param 
                        xmlns:xsd="http://www.w3.org/2001/XMLSchema" 
                        name="mediaType" 
                        style="query" 
                        type="xsd:string">
                    <doc>...</doc>
                    </param>
                    ...
                </request>
                <response> //@\ding{209}@//
                    <representation 
                        xmlns:sns="http://api.spreadshirt.net"
                        element="sns:user" 
                        status="200" 
                        mediaType="application/xml">
                    <doc title="Success"/>
                    </representation>
                    <fault status="500" mediaType="text/plain">
                        <doc title="Internal Server Error"/>
                    </fault>
                    ...
                    }
                ...
\end{lstlisting}
%\end{minipage}

%% ToDo: verfeinern der Beschreibung, was ist erlaubt, was nicht?
Die Datei beginnt mit der Angabe der \textsc{Xml}-Deklaration \ding{202}.
Die Attribute des Wurzelknotens \texttt{<application>} enthalten \emph{namespace} Definitionen, u. a. auch den der verwendeten \textsc{Wadl}-Spezifikation \ding{203}.
Innerhalb des \texttt{<grammars>} Elements werden die benutzten \textsc{Xml} Schemas angegeben \ding{204}. 
Um die Ressourcen der Webanwendung ansprechen zu können wird noch die Angabe der Basisadresse benötigt \ding{205}. 
Innerhalb des \texttt{<resources>} Elements findet sich die Beschreibung der einzelnen Ressourcen. Diese sind gekennzeichnet, durch eine zur Basisadresse relativen \gls{URI} \ding{206}. In \texttt{\{\ldots\}} eingeschlossene Teile einer \gls{URI}, werden durch den Wert des gleichnamigen \emph{request} Parameters ersetzt um die \gls{URI} zu bilden (generative \glspl{URI}).
Im Folgenden werden die von der Ressource unterstützten \textsc{Http}-Methoden beschrieben \ding{207}, deren Anfrageparameter \texttt{<request>} \ding{208}, sowie die möglichen Ausgaben der jeweiligen Methode \texttt{<response>} \ding{209}.

Die Dokumentations-Tags \texttt{<doc>} sind für alle \textsc{Xml}-Elemente optional.
Um das Listing nicht unnötig zu verlängern wurden die schließenden \emph{Tags} weggelassen.

\Cref{fig:wadlstructure} zeigt die Struktur einer \textsc{Wadl}-Datei.

\newpage

\begin{figure}
    \centering
    \resizebox{!}{\textwidth}{
        \begin{tikzpicture}[
    level 1/.style={
        font=\large,
        sibling distance=4cm,
        level distance=5cm
    },
    level 2/.style={
        font=\normalsize,
        level distance=4.5cm,
        sibling distance=7cm
    },
    level 3/.style={
        font=\small,
        level distance=4cm,
        sibling distance=2cm
    },
    level 4/.style={
        level distance=3cm,
        sibling distance=2cm
    }
    edge from parent/.style={
        draw=black,
        shorten >=5mm,
        shorten <=5mm
    },
    edgeText/.style={
        sloped,
        near end,
        above
    },    
    centeredEdgeText/.style={
        sloped,
        above
    },                
    every two node part/.style={
        color=magenta,
        font=\scriptsize,
        align=left
    },
    element/.style={
        rectangle,
        rounded corners,
        draw=gray,
        very thick
    },
    splitElement/.style={
        rectangle,
        rounded corners,
        draw=gray,
        very thick,
        rectangle split, 
        rectangle split parts=2
    }
]
\node {\Large \textbf{application}}
child {
    node [splitElement] {
        doc
        \nodepart{two}
        xml:lang\\
        title
    }
    edge from parent node[edgeText] {0..*};
}
child {
    node [element] {grammars}
    child [level distance = 2.5cm] {
        node [splitElement] {
            include
            \nodepart{two}
            href
        }
        edge from parent node[edgeText] {0..*};
    }
    edge from parent node[edgeText] {0..1};
}
child {
    node [splitElement] {
        resources
        \nodepart{two}
        base
        }
    child [sibling distance=6cm] {
        node [splitElement] {
            resource\_type
            \nodepart{two}
            id
        }
        child {
            node [splitElement] {
                doc
                \nodepart{two}
                \ldots
            }
            edge from parent node [edgeText] {0..*};
        }
        child {
            node [splitElement] {
                param
                \nodepart{two}
                id\\
                name\\
                style\\
                type\\
                default\\
                path\\
                required\\
                repeating\\
                fixed
            }
            child {
                node [splitElement] {
                    option
                    \nodepart{two}
                    value
                }
                child {
                    node [splitElement] {
                        doc
                        \nodepart{two}
                        \ldots
                    }
                    edge from parent node [edgeText] {0..*};
                }
            }
            child {
                node [splitElement] {
                    link
                    \nodepart{two}
                    resource\_type\\
                    rel\\
                    rev\\
                    location\\
                    index
                } 
                edge from parent node [edgeText] {0..*};
            }
            edge from parent node [edgeText] {0..*};
        }
        child {
            node [splitElement] {
                method
                \nodepart{two}
                \ldots
            }
            edge from parent node [edgeText] {0..*};
        }
        edge from parent node [edgeText] {0..*};
    }
    child {
        node [splitElement] {
            resource
            \nodepart{two}
            id\\
            path\\
            type\\
            queryType
            }
        child {
            node [splitElement] {
                doc
                \nodepart{two}
                \ldots
            }
            edge from parent node[edgeText] {0..*};       
        }
        child  {
            node [splitElement] {
                param
                \nodepart{two}
                \ldots
                }
            edge from parent node[edgeText] {0..*};                    
        }
        child {
            node [splitElement] {
                method
                \nodepart{two}
                \ldots
            }
            edge from parent node[edgeText] {0..*};                    
        }
        child {
            node [splitElement] {
                resource
                \nodepart{two}
                \ldots
            }
            edge from parent node[edgeText] {0..*};
        }
        edge from parent node [edgeText] {0..*};                                
    }
    edge from parent node[edgeText] {0..1};
}
child {
    node [element] {resource\_type}
    edge from parent node [edgeText] {0..*};
}            
child [level distance = 6cm] {
    node [element] {method}
    child [sibling distance = 3cm] {
        node [splitElement] {
            method
            \nodepart{two}
            href
        }
        edge from parent node [centeredEdgeText] {reference};
    }
    child [sibling distance = 3cm, level distance=6cm] {
        node [splitElement] {
            method
            \nodepart{two}
            name\\
            id
        }
        child [sibling distance = 2.5cm] {
            node [element] {request}
            child {
                node [element, rectangle split, rectangle split parts=2, align=left] {
                    representation
                    \nodepart{two}
                    \color{green!50!black}
                    nur PUT \& POST\\
                    \ldots
                }
                edge from parent node [edgeText] {0..*};                 
            }
            child {
                node [splitElement] {
                    doc
                    \nodepart{two}
                    \ldots
                }
                edge from parent node [edgeText] {0..*};
            }
            child {
                node [splitElement] {
                    param
                    \nodepart{two}
                    \ldots
                }
                edge from parent node [edgeText] {0..*};
            }
            edge from parent node [edgeText] {0..*};
        }
        child {
            node [splitElement] {
                doc
                \nodepart{two}
                \ldots
            }
            edge from parent node [edgeText] {0..*};
        }
        child [sibling distance = 2.5cm, level distance = 7cm] {
            node [element] {response}
            child {
                node [splitElement] {
                    doc
                    \nodepart{two}
                    \ldots
                }
                edge from parent node [edgeText] {0..*};
            }
            child {
                node [splitElement] {
                    representation
                    \nodepart{two}
                    \ldots
                }
                edge from parent node [edgeText] {0..*};
            }
            child {
                node [splitElement] {
                    fault
                    \nodepart{two}
                    \ldots
                }
                edge from parent node [edgeText] {0..*};
            }
            child {
                node [splitElement] {
                    param
                    \nodepart{two}
                    \color{green!50!black}                    
                    style=header
                    \\
                    \ldots
                }
                edge from parent node [edgeText] {0..*};
            }
            edge from parent node [edgeText] {0..*};
        }
        edge from parent node [centeredEdgeText] {declaration};
    }
    edge from parent node [edgeText] {0..*};
}           
child {
    node [element] {representation}
    child [sibling distance = 3cm,level distance=2.5cm] {
        node [splitElement] {
            representation
            \nodepart{two}
            href
        }
        edge from parent node [centeredEdgeText] {reference};
    }
    child [sibling distance = 3cm, level distance=3.5cm] {
        node [splitElement] {
            representation
            \nodepart{two}
            id\\
            mediaType\\
            element\\
            profile []\\
            status []
        }
        child {
            node [splitElement] {
                param
                \nodepart{two}
                \ldots
            }
            edge from parent node [edgeText] {0..*};
        }
        edge from parent node [centeredEdgeText] {definition};
    }
    edge from parent node [edgeText] {0..*};
}           
child {
    node [splitElement] {
        fault
        \nodepart{two}
        \color{green!50!black}
        entspricht \emph{representation}
    }
    edge from parent node [edgeText] {0..*};
}
;
\end{tikzpicture}  
    }
    \caption{Struktur einer \textsc{Wadl}-Datei, nach Kapitel 2 \cite{hadleyWADL}}
    \label{fig:wadlstructure}
\end{figure}
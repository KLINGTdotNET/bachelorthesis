\subsection{XML Schema Description (XSD)}
\label{sec:xsd}

\emph{XML Schema Description} ist ein stark erweiterte Nachfolger der \emph{DTD} (Document Type Definition), derzeit spezifiert in Version 1.1 \cite{XMLSchema11Specification}. 
Die Syntax von \emph{XSD} ist XML, damit ist die Schemabeschreibung ebenfalls ein gültiges XML-Dokument. Als Dateiendung wird üblicherweise \texttt{.xsd} verwendet.

Die Hauptmerkmale von XSD sind nach \cite[Kapitel 3.2][]{taxonomyXMLSchema}
, die folgenden:
\begin{compactitem}
    \item Komplexe Typen (strukturierter Inhalt)
    \item anonyme Typen (besitzen kein \texttt{type}-Attribut)
    \item Modellgruppen
    \item Ableitung durch Erweiterung oder Einschränkung (\enquote{derivation by extension/restriction})
    \item Definition von abstrakten Typen
    \item Integritätsbedingungen (\enquote{integrity constraints}):\\
        \emph{unique}, \emph{keys} und \emph{keyref}, dies entspricht den \emph{unique-}, \emph{primary-} und \emph{foreign}-keys aus dem Bereich der Datenbanken        
\end{compactitem}

\begin{lstlisting}[
    language=XML,
    caption=Beginn der XSD-Datei für die Spreadshirt-API,
    label=lst:xsdIntro
    ]
<?xml version="1.0" encoding="UTF-8" standalone="yes"?> //@\ding{202}@//
<xs:schema xmlns:xs="http://www.w3.org/2001/XMLSchema"  targetNamespace="http://api.spreadshirt.net" version="1.0" elementFormDefault="qualified"> //@\ding{203}@//
    <xs:import namespace="http://www.w3.org/1999/xlink" schemaLocation="xlink.xsd"/> //@\ding{204}@//
    ...
\end{lstlisting}

Eine XSD-Datei beginnt wie jede XML-Datei mit der XML-Deklaration \ding{202}.

Das Wurzelelement der Schemadefinition zeigt \ding{203}. 
Das Attribut \emph{xmlns:xs= "http://www.w3.org/2001/XMLSchema"} führt den Namespace-Prefix \emph{xs} ein und gibt außerdem an, dass die Elemente und vordefinierten Datentypen (siehe \cref{fig:xsddatatypes}) aus dem Namensraum \emph{http://www.w3.org/2001/XMLSchema} verwendet werden. Durch das Attribut \emph{targetNamespace} wird der Namensraum der Elemente festgelegt, die in dieser Schemadefinition definiert werden. \emph{Version} gibt die XSD-Version an.
Der Wert des Attributs \emph{elementFormDefault} gibt an, ob Elemente des Schemas den \emph{targetNamespace} explizit angeben müssen (\enquote{qualified}) oder ob dies implizit geschieht (\enquote{unqualified}), die Angabe ist optional.

Externe Schemadefinitionen lassen sich unter Angabe des Namensraumes und einer \gls{URI} zu der XSD-Datei einbinden \ding{204}.

XML Schema Description erlaubt die Definition von simplen Typen (\enquote{SimpleType}) und Typen mit strukturiertem Inhalt (\enquote{ComplexType}).

\begin{lstlisting}[
    language=XML,
    caption=Beispiel für einen SimpleType anhand des Type \enquote{unit} der Spreadshirt-API,
    label=lst:xsdExampleUnit
]
<xs:simpleType name="unit">
    <xs:restriction base="xs:string"> //@\ding{202}@//
        <xs:enumeration value="mm"/> //@\ding{203}@//
        <xs:enumeration value="px"/> //@\ding{203}@//
    </xs:restriction>
</xs:simpleType>
\end{lstlisting}

\emph{SimpleType}-Definitionen dienen zur Beschreibung einfacher Typen wie \emph{Enumeratoren}, oder \emph{Listen} für Daten eines primitiven Typs. Ein Beispiel für die Definition eines Enumerators durch einen SimpleType zeigt \Cref{lst:xsdExampleUnit}. Der Basisdatentyp des Enumerators wird dabei durch die Angabe des Attributs \emph{base} \ding{202} festgelegt. Zuordnung von Werten zu dem Enumerator zeigt \ding{203}.

Durch einen SimpleType definierte Listen sind durch Leerzeichen separierte Strings, sie werden meist für den Wert eines Attributes einer XML-Datei verwendet. 

\begin{lstlisting}[
    language=XML,
    caption=Beispiel für einen Listentyp definiert duch einen SimpleType,
    label=lst:xsdListExample    
]
<xs:simpleType name=colors>
    <xs:list itemType="xs:string"/>
</xs:simpleType>
\end{lstlisting}

\begin{lstlisting}[
    language=XML,
    caption=Beispielinstanz für Typ aus \Cref{lst:xsdListExample},
    label=lst:xsdListExampleInstance
]
<test>red green blue</test>}
\end{lstlisting}

Die Definition eines strukturierten Typs zeigt \Cref{lst:xsdExampleAbstractList}.

\begin{lstlisting}[
    language=XML, 
    caption=Beispiel für eine Schemabeschreibung mit XSD anhand des \enquote{abstractList}-Typs der Spreadshirt-API,
    label=lst:xsdExampleAbstractList
    ]
<xs:complexType name="abstractList" abstract="true"> //@\ding{202}@//
    <xs:sequence> //@\ding{203}@//
        <xs:element minOccurs="0" //@\ding{204}@// name="facets"> //@\ding{205}@//
            <xs:complexType> //@\ding{206}@//
                <xs:sequence>
                    <xs:element xmlns:tns="http://api.spreadshirt.net" minOccurs="0" maxOccurs="unbounded" ref="tns:facet" //@\ding{207}@// />
                </xs:sequence>
            </xs:complexType>
        </xs:element>
    </xs:sequence>
    <xs:attribute xmlns:xlink="http://www.w3.org/1999/xlink" ref="xlink:href"/> //@\ding{208}@//
    <xs:attribute type="xs:long" name="offset"/> //@\ding{209}@//
    <xs:attribute type="xs:string" name="query"/>        
    ...
</xs:complexType>
\end{lstlisting}

Das \emph{ComplexType}-Tag \ding{202} umschließt die Definiton des strukturierten Typs. XML Schema Description erlaubt das definieren von abstrakten Typen, nur Ableitungen davon dürfen als Instanzen in einem Dokument auftreten. Abgeleitete Typen dürfen dabei den abstrakten Typ \emph{erweitern} o. \emph{einschränken} (\enquote{derivation by extension/restriction}).

Mit \emph{Reihenfolgeindikatoren} \ding{203} kann die Ordnung von Elementen festgelegt werden. Elemente unterhalb eines \emph{Sequence}-Tags dürfen nur in der Abfolge auftreten in der sie unterhalb des Sequence-Tag definiert worden sind. Das \emph{All}-Tag \ding{205} hingegen erlaubt das auftreten ohne festgelegte Reihenfolge. Der Reihenfolgeindikator \emph{Choice} erlaubt das auftreten nur eines der Elemente die unterhalb dieses Tags vorkommen.

Durch die optionale Angabe von \emph{Häufigkeitsindikatoren} \ding{204} kann festgelegt werden wie oft ein Element an der definierten Stelle vorkommen darf. Entfällt dies, entspricht der Wert von \emph{min-, maxOccurs} "1", das heisst das Element darf genau einmal an dieser Stelle vorkommen.

Elemente einer XML-Datei werden durch das gleichnamige \emph{Element} \ding{205} im XSD definiert. Ein Element benötigt die Angabe eines Namens und Typs. Die Angabe des Typs kann dabei als Referenz auf die Typdefinition \ding{207} oder als Definition unterhalb des Element-Tags erfolgen \ding{206}.

Attribute eines XML-Tags werden durch das \emph{Attribute}-Element definiert. Dies geschieht durch Angabe von Name und Typ \ding{208} oder durch eine Referenz auf eine Attributdefinition \ding{207}.

Referenzen haben die Form \emph{Namensraumbezeicher}:\emph{Elementname}. Wobei mit Elementname jedes Element der Schemabeschreibung gemeint ist, welches ein \emph{name}-Attribut besitzt. Der konkrete Namensraum eines solchen Bezeichners wird vorher mit der Angabe eines Attributes in dieser Form eingführt:

\[  
    \underbrace{xmlns}_{\tiny \text{XML-Namespace}}:\underbrace{tns}_{\tiny \text{Namensraumbzeichner}}=\underbrace{"http://api.spreadshirt.net"}_{\tiny \text{Konkreter Namensraum}}
\]


%
% todo: xml schema genauer beschreiben, relaxng entfernen --: Beispiel fuer Instanz und Definition des Typs!
%

\newpage

\begin{figure}[!t]
    \centering
    \tikzstyle{blueBox}=[
        rectangle,
        fill={blue!15},
        draw,
        font=\sffamily
    ]      
    \tikzstyle{grayBox}=[
        rectangle,
        fill=lightgray,
        text=black,
        font=\sffamily,
        draw
    ]
    \tikzstyle{violetBox}=[
        rectangle,
        fill=violet,
        text=white,
        font=\sffamily,
        draw
    ]
    \tikzstyle{greenBox}=[
        rectangle,
        fill=green!50,
        text=black,
        font=\sffamily,
        draw
    ]
    \tikzstyle{derivedFromList}=[
        dashed,
        cyan
    ]
    \resizebox{!}{0.92\textheight} {
        \begin{minipage}[b]{0.45\linewidth}
        \rotatebox{90}{
        \resizebox{!}{\textwidth}{
                \begin{tikzpicture}[
    level distance=1.1cm,
    level 1/.style={sibling distance=4cm},
    level 2/.style={sibling distance=2cm},
    level 3/.style={sibling distance=2.5cm}
  ]
  \node (root) [violetBox] {anyType}
    [edge from parent fork down]
    child {node[grayBox] {all complex types}
        edge from parent[loosely dashed, magenta]
    }
    child {node[violetBox] {anySimpleType}
            child {node[blueBox] {duration}}
            child {node[blueBox] {dateTime}}
            child {node[blueBox] {time}}
            child {node[blueBox] {date}}
            child {node[blueBox] {gYearMonth}}
            child {node[blueBox] {gYear}}
            child {node[blueBox] {gMonthDay}}
            child {node[blueBox] {gDay}}
            child {node[blueBox] {gMonth}}
            child {
                child [sibling distance = 3cm]{
                    child {node[blueBox] {string}
                        child {node[greenBox] {normalizedString}}
                        child {node[greenBox] {token}
                            child {node[greenBox] {language}}
                            child {node[greenBox] {Name}
                                child {node[greenBox] {NCName}
                                    child {node[greenBox] {ID}}
                                    child {node[greenBox] {IDREF}
                                        child {node[greenBox] {IDREFS}
                                            edge from parent[derivedFromList]
                                        }
                                    }
                                    child {node[greenBox] {ENTITY}
                                        child {node[greenBox] {ENTITIES}
                                            edge from parent[derivedFromList]
                                        }
                                    }
                                }
                            }
                            child {node[greenBox] {NMTOKEN}
                                child  {node [greenBox] {NMTOKENS}
                                    edge from parent[derivedFromList]
                                }
                            }
                        }
                    }
                }
                child {node[blueBox] {boolean}}
                child {node[blueBox] {base64Binary}}
                child {node[blueBox] {hexBinary}}
                child {node[blueBox] {float}}
                child [sibling distance = 3cm] {
                    child {node[blueBox] {decimal}
                        child [sibling distance = 4cm] {node[greenBox] {integer}
                            child {node[greenBox] {nonPositiveInteger}
                                child {node[greenBox] {negativeInteger}}
                            }
                            child {node[greenBox] {long}
                                child {node[greenBox] {int}
                                    child {node[greenBox] {short}
                                        child {node[greenBox] {byte}}
                                    }
                                }
                            }
                            child {node[greenBox] {nonNegativeInteger}
                                child {node[greenBox] {unsignedLong}
                                    child {node[greenBox] {unsignedInt}
                                        child {node[greenBox] {unsignedShort}
                                            child {node[greenBox] {unsignedByte}}
                                        }
                                    }
                                }
                                child {node[greenBox] {positiveInteger}}
                            }
                        }
                    }
                }
                child {node[blueBox] {double}}
                child {node[blueBox] {anyURI}}
                child {node[blueBox] {QName}}
                child {node[blueBox] {NOTATION}}
            }
    };
\end{tikzpicture}
            }
        }
        \end{minipage}
        \hspace{12pt}
        \begin{minipage}[b]{0.45\linewidth}
        \rotatebox{90}{
            \resizebox{!}{0.3\textwidth}{
                \begin{tikzpicture}[framed]
    \node (title) [font=\bfseries] {Legende:};
    \node (base) [violetBox, right = of title] {Basis Typ};
    \node (primitive) [blueBox, right = of base] {Primitiver Typ};
    \node (derived) [greenBox, right = of primitive] {Abgeleiteter Typ};
    \node (complex) [grayBox, right = of derived] {Komplexer Typ};
    \node (d1) [below = of base] {};
    \node (d2) [below = of primitive] {}
        edge [] node[swap, align=center]{Abgeleitet durch\\Einschränkung} (d1);
    \node (d3) [below = of derived] {};
    \node (d4) [below = of complex] {}
        edge [dashed, cyan] node[swap, align=center]{von Liste\\abgeleitet} (d3);
    \node (d5) [below = of d2] {};
    \node (d6) [below = of d3] {}
        edge [loosely dashed, magenta] node[swap, align=center]{Abgeleitet durch\\Erweiterung/Einschränkung} (d5);
\end{tikzpicture}
            }
        }
        \end{minipage}
    }
    \caption{vordefinierte XSD Datentypen nach \cite{XMLSchema11Specification} Kapitel 3}
    \label{fig:xsddatatypes}
\end{figure}
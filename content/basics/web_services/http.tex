\section{HTTP}
\label{sec:http}

\thesisDefinition{HTTP}{
Das \emph{Hypertext Transfer Protocol (HTTP)} ist allgemeines und \emph{zustandsloses} Protokoll, zur Übertragung von Daten über ein Netzwerk, was durch Erweiterung seiner Anfragemethoden, Statuscodes und Header für viele unterschiedliche Anwendungen verwendet werden kann (\cite[][Abstract]{rfc2616}).
}

HTTP arbeitet auf der Anwendungsschicht\footnote{OSI-Modell Level 7} und ist somit unabhängig von dem zum Datentransport eingesetzten Protokoll. 
Über eindeutige \glspl{URI} werden HTTP-Ressourcen angesprochen. Dabei sendet ein \emph{Client} eine Anfrage (\emph{request}) und erhält daraufhin vom Server eine Antwort (\emph{response}). Anfrage und Antwort stellt dabei eine HTTP-Nachricht dar, die aus den zwei Elementen \emph{Header} und \emph{Body} besteht. Letzterer trägt die Nutzdaten und kann, je nach verwendeter HTTP-Methode, auch leer sein.

\citetitle{rfc2616} (\cite{rfc2616}) definiert einige HTTP-Methoden, wobei die gebräuchlichsten die folgenden sind:
\begin{compactitem}
    \item GET
    \item PUT
    \item POST
    \item DELETE
\end{compactitem}

Sie gehören zur Gruppe der sogenannten \emph{CRUD}-Methoden (Create, Receive, Update, Delete) .

\emph{Header} und \emph{Body} bilden die grundlegenden Elemente einer HTTP-Nachricht. 
Eine Nachricht kann je nach verwendeter HTTP-Methode auch nur aus einem Header bestehen.

\subsection{Header}
\label{sec:http-header}

Ein Header einer HTTP-Nachricht besteht aus einer \emph{Request Line} (erste Zeile des Headers) und einer Menge von Schlüssel-Wert Paaren. \Cref{lst:headGETrequest} zeigt einen Beispiel Header für die GET Anfrage auf die Spreadshirt-API Ressource:\\
\texttt{http://api.spreadshirt.net/api/v1/locales}.

\begin{lstlisting}[
    label=lst:headGETrequest,
    caption=HTTP-Header von GET Request auf Spreadshirt-API Ressource \texttt{http://api.spreadshirt.net/api/v1/locales}
]
GET //@\ding{202}@// /api/v1/locales //@\ding{203}@// HTTP/1.1 //@\ding{204}@//
User-Agent: curl/7.29.0 //@\ding{205}@//
Host: api.spreadshirt.net //@\ding{206}@//
Accept: */* //@\ding{207}@//
\end{lstlisting} 

\begin{compactitem}
    \item[\ding{202}] Angabe der HTTP-Methode
    \item[\ding{203}] Ressource
    \item[\ding{204}] verwendete HTTP-Version
    \item[\ding{205}] \emph{User-Agent}, Angabe zum Client-System das die Anfrage versendet
    \item[\ding{206}] \emph{Host}, Server der die Anfrage erhält und der die Ressource \ding{203} verwaltet
    \item[\ding{207}] Angabe von \emph{Content-Types} die der Client als Antwort akzeptiert, in diesem Fall eine \emph{Wildcard}, also alle Typen sind als Antwort erlaubt
\end{compactitem}

Nachfolgend die \emph{Response} auf den soeben beschrieben \emph{Request}.

\begin{lstlisting}[
    label=lst:headGETresponse,
    caption=HTTP-Header von GET Response aus Spreadshirt-API Ressource \texttt{http://api.spreadshirt.net/api/v1/locales}
]
HTTP/1.1 200 OK //@\ding{202}@//
Expires: Tue, 20 Aug 2013 19:05:25 GMT
Content-Language: en-gb
Content-Type: application/xml;charset=UTF-8 //@\ding{203}@//
X-Cache-Lookup: MISS from fish07:80
X-Server-Name: mem1
True-Client-IP: 88.79.226.66
Date: Tue, 20 Aug 2013 07:20:25 GMT
Content-Length: 1659
Connection: keep-alive
\end{lstlisting}

\begin{compactitem}
    \item[\ding{202}] \emph{Response Line}, Angabe der HTTP-Version am Anfang und danach der HTTP-Statuscode mit Kurzbeschreibung
    \item[\ding{203}] Angabe des Content-Types des Bodys der Nachricht
\end{compactitem}

Welche Einträge der Header einer HTTP-Nachricht letztendlich enhtält, ist abhängig von der Implementierung des Clients oder Servers und es können auch jederzeit eigene Einträge, die nicht in der HTTP-Spezifikation enthalten sind, hinzugefügt werden.

\subsection{Body}
\label{sec:http-body}

Der Body enthält die eigentlichen Nutzdaten. Deren Format wird mit dem \emph{Content-Type} Eintrag des Headers angegeben. \Cref{lst:bodyGETresponse} zeigt den Body der \emph{response} von \cref{lst:headGETresponse} in \gls{XML} Format (siehe \cref{sec:xml}).

\begin{lstlisting}[
    language=XML,
    label=lst:bodyGETresponse,
    caption=HTTP-Body von GET Response aus Spreadshirt-API Ressource \texttt{http://api.spreadshirt.net/api/v1/locales}
]
<?xml version="1.0" encoding="UTF-8" standalone="yes"?>
<locales xmlns:xlink="http://www.w3.org/1999/xlink" xmlns="http://api.spreadshirt.net" xlink:href="http://api.spreadshirt.net/api/v1/locales" offset="0" limit="50" count="16" sortField="default" sortOrder="default">
    <locale xlink:href="http://api.spreadshirt.net/api/v1/locales/de_DE" id="de_DE"/>
    <locale xlink:href="http://api.spreadshirt.net/api/v1/locales/fr_FR" id="fr_FR"/>
    <locale xlink:href="http://api.spreadshirt.net/api/v1/locales/en_GB" id="en_GB"/>
    <locale xlink:href="http://api.spreadshirt.net/api/v1/locales/nl_NL" id="nl_NL"/>
    <locale xlink:href="http://api.spreadshirt.net/api/v1/locales/it_IT" id="it_IT"/>
    <locale xlink:href="http://api.spreadshirt.net/api/v1/locales/no_NO" id="no_NO"/>
    <locale xlink:href="http://api.spreadshirt.net/api/v1/locales/se_SE" id="se_SE"/>
    <locale xlink:href="http://api.spreadshirt.net/api/v1/locales/es_ES" id="es_ES"/>
    <locale xlink:href="http://api.spreadshirt.net/api/v1/locales/en_EU" id="en_EU"/>
    <locale xlink:href="http://api.spreadshirt.net/api/v1/locales/en_IE" id="en_IE"/>
    <locale xlink:href="http://api.spreadshirt.net/api/v1/locales/dk_DK" id="dk_DK"/>
    <locale xlink:href="http://api.spreadshirt.net/api/v1/locales/pl_PL" id="pl_PL"/>
    <locale xlink:href="http://api.spreadshirt.net/api/v1/locales/fi_FI" id="fi_FI"/>
    <locale xlink:href="http://api.spreadshirt.net/api/v1/locales/de_AT" id="de_AT"/>
    <locale xlink:href="http://api.spreadshirt.net/api/v1/locales/fr_BE" id="fr_BE"/>
    <locale xlink:href="http://api.spreadshirt.net/api/v1/locales/nl_BE" id="nl_BE"/>
</locales>
\end{lstlisting}

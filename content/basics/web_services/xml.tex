\subsection{XML}
\label{sec:xml}

\thesisDefinition{\gls{XML}}{
Die \enquote{Extensible Markup Language}, kurz \gls{XML}, ist eine Auszeichnungssprache (\enquote{Markup Language}), die eine Menge von Regeln beschreibt um Dokumente in einem mensch- und maschinenlesbaren Format zu kodieren \cite{XML10Specification}.
}

Obwohl das Design von \gls{XML} auf Dokumente ausgerichtet ist, wird es häufig für die Darstellung von beliebigen Daten benutzt \cite{wiki:xml}, z.B. um diese für die Übertragung zu serialisieren.

Eine valide \gls{XML}-Datei beginnt mit der \gls{XML}-Deklaration \ding{202}. Diese enthält Angaben über die verwendete \gls{XML}-Spezifikation und die Kodierung der Datei. 
Im Gegensatz zu gewöhnlichen Tags wird dieses mit \texttt{<?} und mit \texttt{?>} beendet. 
Danach folgen beliebig viele baumartig geschachtelte \emph{Elemente} mit einem Wurzelelement \ding{203}. Die Elemente können Attribute enthalten und werden, wenn sie kein leeres Element sind \ding{204}, von einem schließenden Tag in der gleichen Stufe abgeschlossen \ding{206}. Nicht leere Zeichenketten als Kindelement sind ebenfalls erlaubt \ding{205}.

Mit Hilfe von \emph{Schemabeschreibungssprachen} (\cref{sec:xmlschema}) kann der Inhalt und die Struktur eines Dokumentes festgelegt und gegen diese validiert werden. Der Begriff \gls{XML} Schema ist mehrdeutig und wird oft auch für eine konkrete Beschreibungssprache, die \enquote{\gls{XML} Schema Definition}, kurz \gls{XSD}, verwendet.

\begin{minipage}{\textwidth}
\begin{lstlisting}[
        language=XML, 
        caption=Die gekürzte Antwort der \textsc{API}-Ressource \texttt{users/{userid}/designs/{designID}} als Beispiel für eine \gls{XML}-Datei
    ][htb]
<?xml version="1.0" encoding="UTF-8" standalone="yes"?> //@\ding{202}@//
<design xmlns:xlink="http://www.w3.org/1999/xlink" 
        xmlns="http://api.spreadshirt.net" 
        ...>//@\ding{203}@//
    <name>tape_recorder</name>
    ...
    <size unit="px">
        <width>3340.0</width>
        <height>3243.0</height>
    </size>
    <colors/>//@\ding{204}@//
    ...
    <created>
        2013-03-30T12:37:54Z //@\ding{205}@//
    </created>
    <modified>2013-04-02T11:13:02Z</modified>
</design>//@\ding{206}@//
\end{lstlisting}
\end{minipage}
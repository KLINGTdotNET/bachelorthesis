\subsection{XML Schema Description (XSD)}
\label{sec:xsd}

\begin{figure}[tb]
    \centering
    \tikzstyle{blueBox}=[
        rectangle,
        fill={blue!15},
        draw,
        font=\sffamily
    ]      
    \tikzstyle{grayBox}=[
        rectangle,
        fill=lightgray,
        text=black,
        font=\sffamily,
        draw
    ]
    \tikzstyle{violetBox}=[
        rectangle,
        fill=violet,
        text=white,
        font=\sffamily,
        draw
    ]
    \tikzstyle{greenBox}=[
        rectangle,
        fill=green!50,
        text=black,
        font=\sffamily,
        draw
    ]
    \tikzstyle{derivedFromList}=[
        dashed,
        cyan
    ]
    \begin{minipage}[b]{0.45\linewidth}
    \rotatebox{90}{
    \resizebox{!}{\textwidth}{
            \begin{tikzpicture}[
    level distance=1.1cm,
    level 1/.style={sibling distance=4cm},
    level 2/.style={sibling distance=2cm},
    level 3/.style={sibling distance=2.5cm}
  ]
  \node (root) [violetBox] {anyType}
    [edge from parent fork down]
    child {node[grayBox] {all complex types}
        edge from parent[loosely dashed, magenta]
    }
    child {node[violetBox] {anySimpleType}
            child {node[blueBox] {duration}}
            child {node[blueBox] {dateTime}}
            child {node[blueBox] {time}}
            child {node[blueBox] {date}}
            child {node[blueBox] {gYearMonth}}
            child {node[blueBox] {gYear}}
            child {node[blueBox] {gMonthDay}}
            child {node[blueBox] {gDay}}
            child {node[blueBox] {gMonth}}
            child {
                child [sibling distance = 3cm]{
                    child {node[blueBox] {string}
                        child {node[greenBox] {normalizedString}}
                        child {node[greenBox] {token}
                            child {node[greenBox] {language}}
                            child {node[greenBox] {Name}
                                child {node[greenBox] {NCName}
                                    child {node[greenBox] {ID}}
                                    child {node[greenBox] {IDREF}
                                        child {node[greenBox] {IDREFS}
                                            edge from parent[derivedFromList]
                                        }
                                    }
                                    child {node[greenBox] {ENTITY}
                                        child {node[greenBox] {ENTITIES}
                                            edge from parent[derivedFromList]
                                        }
                                    }
                                }
                            }
                            child {node[greenBox] {NMTOKEN}
                                child  {node [greenBox] {NMTOKENS}
                                    edge from parent[derivedFromList]
                                }
                            }
                        }
                    }
                }
                child {node[blueBox] {boolean}}
                child {node[blueBox] {base64Binary}}
                child {node[blueBox] {hexBinary}}
                child {node[blueBox] {float}}
                child [sibling distance = 3cm] {
                    child {node[blueBox] {decimal}
                        child [sibling distance = 4cm] {node[greenBox] {integer}
                            child {node[greenBox] {nonPositiveInteger}
                                child {node[greenBox] {negativeInteger}}
                            }
                            child {node[greenBox] {long}
                                child {node[greenBox] {int}
                                    child {node[greenBox] {short}
                                        child {node[greenBox] {byte}}
                                    }
                                }
                            }
                            child {node[greenBox] {nonNegativeInteger}
                                child {node[greenBox] {unsignedLong}
                                    child {node[greenBox] {unsignedInt}
                                        child {node[greenBox] {unsignedShort}
                                            child {node[greenBox] {unsignedByte}}
                                        }
                                    }
                                }
                                child {node[greenBox] {positiveInteger}}
                            }
                        }
                    }
                }
                child {node[blueBox] {double}}
                child {node[blueBox] {anyURI}}
                child {node[blueBox] {QName}}
                child {node[blueBox] {NOTATION}}
            }
    };
\end{tikzpicture}
        }
    }
    \end{minipage}
    \hspace{12pt}
    \begin{minipage}[b]{0.45\linewidth}
    \rotatebox{90}{
        \resizebox{!}{0.3\textwidth}{
            \begin{tikzpicture}[framed]
    \node (title) [font=\bfseries] {Legende:};
    \node (base) [violetBox, right = of title] {Basis Typ};
    \node (primitive) [blueBox, right = of base] {Primitiver Typ};
    \node (derived) [greenBox, right = of primitive] {Abgeleiteter Typ};
    \node (complex) [grayBox, right = of derived] {Komplexer Typ};
    \node (d1) [below = of base] {};
    \node (d2) [below = of primitive] {}
        edge [] node[swap, align=center]{Abgeleitet durch\\Einschränkung} (d1);
    \node (d3) [below = of derived] {};
    \node (d4) [below = of complex] {}
        edge [dashed, cyan] node[swap, align=center]{von Liste\\abgeleitet} (d3);
    \node (d5) [below = of d2] {};
    \node (d6) [below = of d3] {}
        edge [loosely dashed, magenta] node[swap, align=center]{Abgeleitet durch\\Erweiterung/Einschränkung} (d5);
\end{tikzpicture}
        }
    }
    \end{minipage}
    \caption{vordefinierte XSD Datentypen nach \cite{XMLSchema11Specification} Kapitel 3}
    \label{fig:xsddatatypes}
\end{figure}

\emph{XML Schema Description} ist ein stark erweiterte Nachfolger der \emph{DTD} (Document Type Definition), derzeit spezifiert in Version 1.1 \cite{XMLSchema11Specification}. 
Die Syntax von \emph{XSD} ist XML, damit ist die Schemabeschreibung ebenfalls ein gültiges XML-Dokument. Als Dateiendung wird üblicherweise \texttt{.xsd} verwendet.
Die Hauptmerkmale von XSD sind nach \cite{taxonomyXMLSchema} % Kapitel 3.2
, die folgenden:
\begin{compactitem}
    \item Komplexe Typen (strukturierter Inhalt)
    \item anonyme Typen (besitzen kein \texttt{type}-Attribut)
    \item Modellgruppen
    \item Ableitung durch Erweiterung oder Einschränkung (\enquote{derivation by extension/restriction})
    \item Definition von abstrakten Typen
    \item Integritätsbedingungen (\enquote{integrity constraints}):\\
        \emph{unique}, \emph{keys} und \emph{keyref}, dies entspricht den \emph{unique-}, \emph{primary-} und \emph{foreign}-keys aus dem Bereich der Datenbanken
\end{compactitem}

Die XSD Spezifikation enthält bereits eine Menge vordefinierter Datentypen, dargestellt in \cref{fig:xsddatatypes}.

% todo: standalone attribute, elementFormDefault?
\begin{lstlisting}[language=XML, caption=Minimalbeispiel für eine Schemabeschreibung mit XSD, label=lst:minimalXSD]
<?xml version="1.0" encoding="UTF-8" standalone="yes"?>
<xsd:schema 
    xmlns:xsd="http://www.w3.org/2001/XMLSchema"
    version="1.0" 
    targetNamespace="myNamespace" 
    elementFormDefault="qualified">
    <xsd:complexType name="product">
        <xsd:sequence>
            <xsd:element name="name" type="xsd:string"/>
            <xsd:element name="price" type="xsd:decimal"/>
            <xsd:element name="description"
                ref="myNamespace:description"/>
        </xsd:sequence>
    </xsd:complexType>
    <xsd:complexType name="description">
        <xsd:all>
            <xsd:element name="title" type="xsd:string">
            <xsd:element name="content" type="xsd:string">
        </xsd:all>
    </xsd:complexType>
</xsd:schema>
\end{lstlisting}

%
% todo: wie detailiert Grundlangen beschreiben?
%

%Komplexe Typen werden durch Elemente vom Typ \texttt{xsd:complexType} definiert, sie dienen zur Definition von XML-Inhalt aus Elementen mit Attributen. Elemente können hierbei Deklarationen oder Referenzen auf Elementdeklarationen sein.

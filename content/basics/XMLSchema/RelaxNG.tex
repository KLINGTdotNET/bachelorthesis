\subsection{RelaxNG}
\label{sec:relaxng}

% Wikipedia, satz abändern
Ebenso wie XSD (siehe \cref{sec:xsd}) ist \emph{Regular Language Description for XML New Generation} eine XML-Schemasprache zur Definition der Struktur von XML-Dokumenten. Schemas werden in \emph{RelaxNG} durch XML-Syntax oder eine eigene, kompaktere nicht-XML Syntax formuliert. Ebenso wie bei \emph{XML Schema} werden Namespaces unterstützt. RelaxNG Schemabeschreibungen verwenden meist \texttt{.rng} als Dateiendung.

Unterschiede zu XML Schema:
\begin{compactitem}
    \item Unterstützung von ungeordneten Inhalten
    \item kompaktere nicht-XML Syntax
    \item \emph{nichtdeterministisches} oder auch \emph{mehrdeutiges} Inhaltsmodell (\cite{RelaxNGVlist} Kapitel 16)% ToDo: Belge, warum, Erklärung
\end{compactitem}

% Beispiel von http://pike.psu.edu/publications/toit05.pdf Seite 17
\begin{lstlisting}[
    language=XML, 
    caption=Minimalbeispiel für eine Schemadefinition in RelaxNG, 
    label=minimalRelaxNG]
<?xml version="1.0" encoding="utf-8"?>
<grammar 
    xmlns="http://relaxng.org/ns/structure/1.0"
    ns="myNamespace">
    <start>
        <ref name="product"/>
    </start>
    <define name="product">
        <oneOrMore>
            <element name="name"/>
                <text/>
            </element>
            <element name="price"/>
                <text/>
            </element>
            <ref name="description"/>
        </oneOrMore>
    </define>
    <define name="description">
        <oneOrMore>
            <element name="title">
                <text/>
            </element>
            <element name="content">
                <text/>
            </element>
        </oneOrMore>
    </define>
</grammar>
\end{lstlisting}

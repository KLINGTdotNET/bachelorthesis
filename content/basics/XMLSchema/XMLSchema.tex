% ToDo: Titel in Schemasprachen ändern?
\section{XML Schemabeschreibungssprachen (XML Schema)}
\label{sec:xmlschema}

\emph{XML Schema} bezeichnet XML-basierte Sprachen mit denen sich Elemente, Attribute und Aufbau eines XML-Dokumentes beschreiben lassen. 
Ein XML-Dokument wird als \emph{valid/gültig} gegenüber einem Schema bezeichnet, falls die Elemente und Attribute dieses Dokumentes die Bedingungen des Schemas erfüllen \cite{taxonomyXMLSchema}.
Neben XSD (siehe \cref{sec:xsd}) und RelaxNG (siehe \cref{sec:relaxng}) existieren noch weitere Schemasprachen, die hier aber aufgrund ihrer geringen Relevanz nicht behandelt werden. Die beiden hier behandelten Schemasprachen bieten den Vorteil selbst XML-Dokumente zu sein, somit können sie durch herkömmliche XML-Tools bearbeitet werden.

w\subsection{XML Schema Description (XSD)}
\label{sec:xsd}

Die \emph{XML Schema Description} ist der stark erweiterte Nachfolger der \emph{DTD} (Document Type Definition), derzeit spezifiert in Version 1.1 \cite{XMLSchema11Specification}. 
Die Syntax von \emph{XSD} ist XML, damit ist die Schemabeschreibung ebenfalls ein gültiges XML-Dokument.
% ToDo: Satz behalten?
Zur Beschreibung der Spreadshirt-API Daten wird dieses Format genutzt...
Die Spreadshirt API liefert XSDs zur Schemabeschreibung aus, deshalb wird die Beschreibungssprache im folgenden detailierter behandelt.

Komplexe Typen werden durch Elemente vom Typ \texttt{xsd:complexType} definiert, sie dienen zur Definition von XML-Inhalt aus Elementen mit Attributen. Elemente können hierbei Deklarationen oder Referenzen auf Elementdeklarationen sein.
Simple Typen \texttt{xsd:simpleType} 

\subsection{RelaxNG}
\label{sec:relaxng}

% Wikipedia, satz abändern
Ebenso wie XSD (siehe \cref{sec:xsd}) ist \emph{Regular Language Description for XML New Generation} eine XML-Schemasprache zur Definition der Struktur von XML-Dokumenten. Schemas werden in \emph{RelaxNG} durch XML-Syntax oder eine eigene, kompaktere nicht-XML Syntax formuliert. Ebenso wie bei \emph{XML Schema} werden Namespaces unterstützt. RelaxNG Schemabeschreibungen verwenden meist \texttt{.rng} als Dateiendung.

Unterschiede zu XML Schema:
\begin{compactitem}
    \item Unterstützung von ungeordneten Inhalten
    \item kompaktere nicht-XML Syntax
    \item \emph{nichtdeterministisches} oder auch \emph{mehrdeutiges} Inhaltsmodell (\cite{RelaxNGVlist} Kapitel 16)% ToDo: Belge, warum, Erklärung
\end{compactitem}

% Beispiel von http://pike.psu.edu/publications/toit05.pdf Seite 17
\begin{lstlisting}[
    language=XML, 
    caption=Minimalbeispiel für eine Schemadefinition in RelaxNG, 
    label=minimalRelaxNG]
<?xml version="1.0" encoding="utf-8"?>
<grammar 
    xmlns="http://relaxng.org/ns/structure/1.0"
    ns="myNamespace">
    <start>
        <ref name="product"/>
    </start>
    <define name="product">
        <oneOrMore>
            <element name="name"/>
                <text/>
            </element>
            <element name="price"/>
                <text/>
            </element>
            <ref name="description"/>
        </oneOrMore>
    </define>
    <define name="description">
        <oneOrMore>
            <element name="title">
                <text/>
            </element>
            <element name="content">
                <text/>
            </element>
        </oneOrMore>
    </define>
</grammar>
\end{lstlisting}

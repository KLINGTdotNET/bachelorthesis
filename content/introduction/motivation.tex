%\section{Motivation}
%\label{sec:motivation}

Die zwei wichtigsten Konstanten in der Anwendungsentwicklung sind laut \parencite{herrington2003code} folgende:
\begin{compactitem}
    \item Die Zeit eines Programmierers ist kostbar.
    \item Programmierer mögen keine langweiligen und repetitiven Aufgaben.
\end{compactitem}
Codegenerierung greift bei beiden Punkten an und kann zu einer Steigerung der \emph{Produktivität} führen, die durch herkömmliches schreiben von Code nicht zu erreichen wäre. 

Änderungen können an zentraler Stelle vorgenommen und durch die Generierung automatisch in den Code übertragen werden, was mit verbesserter \emph{Wartbarkeit} und erhöhter Effizienz einhergeht.
Die gewonnenen Freiräume kann der Entwickler nutzen um sich mit den grundlegenden Herausforderungen und Problemen seiner Software zu beschäftigen.

Durch die Festlegung eines Schemas für Variablennamen und Funktionssignaturen wird eine hohe \emph{Konsistenz} über die gesamte Codebasis hinweg erreicht.
Diese Einheitlichkeit vereinfacht auch die Nutzung des Generats, da beispielsweise nicht mit Überraschungen bei den verwendeten Bezeichnern zu rechnen ist.

% Als Eingabe für den Generator dient ein \emph{abstraktes Modell} des betreffenden Geschäftsbereiches. Die Erstellung eines solchen Modells vertieft das Verständnis des Entwicklers für das Geschäftsfeld und gibt gleichzeitig Spezialisten aus dem Fachbereich die Möglichkeit Fragestellungen anhand dieses Modells zu formulieren.

%Um die immer kürzer werdenden Entwicklungszyklen einhalten zu können, kann durch Codegenerierung die nötige Effizienzsteigerung geleistet werden.

Zusätzlich zu dem bereits genannten allgemeinen Nutzen einer Codegenerierungslösung, entstehen für Spreadshirt noch die folgenden Vorteile:
\begin{compactitem}
    \item Vereinheitlichung bestehender Implementierungen in Form der generierten Bibliothek
    \item Kapselung der Authentifizierung durch Integration in Client-Bibliothek (\cref{sec:api_auth})
    \item Erleichterung der \gls{API}-Nutzung für externe Entwickler
\end{compactitem}

\section{Anforderungen an die Client-Bibliothek}
\label{item:requirements}

\begin{compactitem}
    \item Austauschbarkeit der Zielsprache
    \item Einfache Bedienbarkeit der Bibliothek
    \item gute Lesbarkeit des erzeugten Codes
    \item größtmögliche Typsicherheit
    \item hohe Testabdeckung der Bibliothek
    \item vollständige Generierung aller Methoden aus der \gls{API}-Beschreibung
\end{compactitem}
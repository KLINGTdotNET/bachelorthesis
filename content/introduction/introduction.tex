\chapter{Einführung}

\chapterQuote{Essentially, all models are wrong, but some are useful.}{George E. P. Box, Norman R. Draper}{1987}{Empirical Model-Building and Response Surfaces. p. 424}

\begin{figure}[htb]
    \centering
    \resizebox{\textwidth}{!}{
        \begin{tikzpicture}[
    	node distance=12mm and 8mm,
    	every node/.style={font=\scriptsize}
    ]
    % Blocks
    \node(abstract)[greyBlock, double copy shadow]{abstrakte\\API Beschreibung};
    \node(parser)[greyBlock, right=of abstract]{Parser};
    \node(APImodel)[greyBlock, right=of parser]{REST-API\\Modell};
    \node(generator)[greyBlock, right=of APImodel]{Codegenerator\\\textbf{Java}};
    \node(languagemodel)[greyBlock, right= of generator]{Sprachenmodell\\\emph{Abstrakter}\\ \emph{Syntaxbaum}};
    \node(infrastructurecode)[greyBlock, above=of generator]{Infrastrukturcode};
    \node(printer)[greyBlock, right= of languagemodel]{Ausgabemodul};
	\node(bib)[greyBlock, right=of printer]{Client-Bibliothek\\\textbf{php}};
	% Lines  
	\path[arrow, ->] (abstract) -- (parser);
	\path[arrow, ->] (parser) -- (APImodel);
    \path[arrow, ->] (APImodel) -- (generator);
    \path[arrow, ->] (infrastructurecode) -- (generator);
	\path[arrow, ->] (generator) -- (languagemodel);
    \path[arrow, ->] (languagemodel) -- (printer);
    \path[arrow, ->] (printer) -- (bib); 
\end{tikzpicture}

    }
    \caption{Aufbau des Generatorsystems}
    \label{fig:generatorstructure}
\end{figure}        

Das Ziel dieser Arbeit ist die Erstellung eines Codegenerators, der aus der abstrakten Beschreibung der Spreadshirt-API eine Client-Bibliothek erstellt.

Der Generator soll eine flexible Wahl der Zielsprache bieten, wobei mit \enquote{Zielsprache} im folgenden die Programmiersprache der erzeugten Bibliothek gemeint ist. 
Für das Bibliotheksdesign ist eine \gls{DSL} (Domain-Specific Language) zu realisieren, mit dem Ziel die Nutzung der \gls{API} zu vereinfachen. 

Als Programmiersprache für den Generator wird \emph{Java} verwendet, als Zielsprache der Bibliothek dient \emph{PHP}. 
%Todo: Template-Engine, falls nicht, entfernen
\sout{Um die gewünschte Flexibilität bezüglich der Zielsprache zu erreichen, wird eine \gls{template-engine} verwendet.}
Eine gute Lesbarkeit, hohe Testabdeckung und größtmögliche Typsicherheit, soweit \emph{PHP} dies zulässt, sind Erfolgskriterien für die zu generierende Bibliothek.

\cref{fig:generatorstructure} stellt den schematischen Aufbau des gewünschten Generators dar.


\section{Spreadshirt}

%\section{Motivation}
%\label{sec:motivation}

Die zwei wichtigsten Konstanten in der Anwendungsentwicklung sind laut \parencite{herrington2003code} folgende:
\begin{compactitem}
    \item Die Zeit eines Programmierers ist kostbar
    \item Programmierer mögen keine langweiligen und repetitiven Aufgaben
\end{compactitem}
Codegenerierung greift bei beiden Punkten an und kann zu einer Steigerung der \emph{Produktivität} führen, die durch herkömmliches schreiben von Code nicht zu erreichen wäre. 

Änderungen können an zentraler Stelle vorgenommen und durch die Generierung automatisch in den Code übertragen werden, was mit verbesserter \emph{Wartbarkeit} einhergeht.
Die gewonnenen Freiräume kann der Entwickler nutzen um sich mit den Grundlegenden Herausforderungen und Problemen seiner Software zu beschäftigen.

Durch die Festlegung eines Schemas für Variablennamen und Funktionssignaturen, wird eine hohe \emph{Konsistenz}, über die gesamte Codebasis hinweg, erreicht.
Diese Einheitlichkeit vereinfacht auch die Nutzung des Generats\footnote{Ergebnis des Codegenerierungsvorgangs}, da beispielsweise nicht mit Überraschungen bei den verwendeten Bezeichnern zu rechnen ist.

Als Eingabe für den Generator dient ein \emph{abstraktes Modell} des betreffenden Geschäftsbereiches. Die Erstellung eines solchen Modells vertieft das Verständnis des Entwicklers für das Geschäftsfeld und gibt gleichzeitig Spezialisten aus dem Fachbereich die Möglichkeit Fragestellungen anhand dieses Modells zu formulieren.

Um die immer kürzer werdenden Entwicklungszyklen einhalten zu können, kann durch Codegenerierung die nötige Effizienzsteigerung geleistet werden.

% todo: allgemeine Motivation warum Codegenerierung ueberhaupt gemacht wird. Hier sollte aber speziell stehen warum gerade bei Spreadshirt Codegenerierung verwendet werden soll. Außerdem sollte anhand eines kurzen Beispiels gezeigt werden wie der Stand vor- und nach der Generierung war.
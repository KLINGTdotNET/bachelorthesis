% Problemstellung
\section{Ziel der Arbeit}

Es ist ein Codegenerator zu erstellen, der aus der abstrakten Beschreibung der \gls{RESTful} Spreadshirt \gls{API} eine Client-Bibliothek erstellt. 

Der Generator soll eine flexible Wahl der Zielsprache bieten, wobei mit \enquote{Zielsprache} im folgenden die Programmiersprache der erzeugten Bibliothek gemeint ist. 
Für das Bibliotheksdesign ist eine \gls{DSL} (Domain-Specific Language) zu realisieren, mit dem Ziel die Nutzung der \gls{API} zu vereinfachen. 

Als Programmiersprache für den Generator wird \emph{Java} verwendet, als Zielsprache der Bibliothek dient \textsc{Php}. Um die gewünschte Flexibilität bezüglich der Zielsprache zu erreichen, wird eine \gls{template-engine} verwendet.

\cref{fig:generatorstructure} stellt den schematischen Aufbau des gewünschten Generators dar.

\begin{figure}[tb]
    \centering
    \resizebox{\textwidth}{!}{
        \begin{tikzpicture}[
    	node distance=12mm and 8mm,
    	every node/.style={font=\scriptsize}
    ]
    % Blocks
    \node(abstract)[greyBlock, double copy shadow]{abstrakte\\API Beschreibung};
    \node(parser)[greyBlock, right=of abstract]{Parser};
    \node(APImodel)[greyBlock, right=of parser]{REST-API\\Modell};
    \node(generator)[greyBlock, right=of APImodel]{Codegenerator\\\textbf{Java}};
    \node(languagemodel)[greyBlock, right= of generator]{Sprachenmodell\\\emph{Abstrakter}\\ \emph{Syntaxbaum}};
    \node(infrastructurecode)[greyBlock, above=of generator]{Infrastrukturcode};
    \node(printer)[greyBlock, right= of languagemodel]{Ausgabemodul};
	\node(bib)[greyBlock, right=of printer]{Client-Bibliothek\\\textbf{php}};
	% Lines  
	\path[arrow, ->] (abstract) -- (parser);
	\path[arrow, ->] (parser) -- (APImodel);
    \path[arrow, ->] (APImodel) -- (generator);
    \path[arrow, ->] (infrastructurecode) -- (generator);
	\path[arrow, ->] (generator) -- (languagemodel);
    \path[arrow, ->] (languagemodel) -- (printer);
    \path[arrow, ->] (printer) -- (bib); 
\end{tikzpicture}

    }
    \caption{Aufbau des Generatorsystems}
    \label{fig:generatorstructure}
\end{figure}        